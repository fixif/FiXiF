\begin{command}[@FWS/FWS]{FWS}
	\desc{Purpose}
Constructor of the FWS class.
A structuration is characterized by an initial realization and a \matlab{way} to transform this realization
	\desc{Syntax}
\matlab{S = FWS(Rini, UYWfun, Rfun, dataFWS, param1Name, param1Value, param2Name, param2Value, ...)}
	\desc{Parameters}
		\begin{tabular}{l@{\ :\ }p{9cm}}
\matlab{S} &  FWS object                                                                                  \\
\matlab{Rini} &  initial realization (FWR object)                                                         \\
\matlab{UYWfun} &  handle to a function that links the parameters to the transformation matrices U,Y and W\\
\matlab{Rfun } &  handle to a function that links the parameters to the new realization                   \\
\matlab{} &  only ONE of these two functions must be provided                                             \\
\matlab{dataFWS} &  cells of extra datas                                                                  \\
\matlab{paramName} &  parameters' names                                                                   \\
\matlab{paramValue} &  initial value for the parameters                                                   \\
		\end{tabular}
	\desc{Description}
This function is called to construct a FWS object.\\
Only one of the two functions \matlab{UYWfun} \matlab{Rfun} must be given (a handle to a function is defined by \matlab{@} + \matlab{name of the function} - see Matlab's documentation for more informations on function's handle).\\
These functions must satisfy the specifications explain in section \ref{sec:FWSclass}.\\
The names and values of each parameter are given by pair.
	\desc{Example}
Let us consider a state-space realization $(A,B,C,D)$.\\
The equivalent realizations are given by the state-space $(T^{-1}AT,T^{-1}B,CT,D)$. This correspond to the following SIF
\begin{equation}
Z =
\begin{pmatrix}
. & . & . \\
. & A_q & B_q \\
. & C_q & D_q
\end{pmatrix}
\end{equation}
and the $\mt{Y}\mt{U}\mt{W}$ transformation with $\mt{U}=T$, $\mt{Y}=\mt{W}=I_{l}$.
Then, to create a state-space structuration, from matrices \matlab{A}, \matlab{B}, \matlab{C} and \matlab{D},
with a parameter \matlab{T}, one should create a UYWfun like
\begin{verbatim}
% UYW function for the classical state-space structuration
function [U,Y,W,cost_flag] = UYW_SS( Rini, paramsValue, dataFWS)
%test if T is singular
if (cond(paramsValue{1})>1e10)
cost_flag=0;
paramsValue{1} = eye(size(paramsValue{1}));
else
cost_flag=1;
end
% compute U,W,Y
Y = eye(0);
W = eye(0);
U = paramsValue{1};
\end{verbatim}
The \matlab{cost\_flag} could return $0$ if the paramsValue proposed is not acceptable (here a non-invertible matrix).\\
Then the structuration is created by
\begin{verbatim}
Rini =  SS2FWR(A,B,C,D);
S = FWS( Rini, @UYW_SS, [], [], 'T', eye(R.n));
\end{verbatim}
(there is no need for a \matlab{dataFWS}).
Even if it is not preferrable, it is also possible to create this FWS with a \matlab{Rfun} function.\\
So a function that creates a new state-space realization from the \matlab{paramsValue} is needed
\begin{verbatim}
% UYW function for the classical state-space structuration
function [R,cost_flag] = Rfun_SS( Rini, paramsValue, dataFWS)
%test if T is singular
if (cond(paramsValue{1})>1e10)
cost_flag=0;
paramsValue{1} = eye(size(paramsValue{1}));
else
cost_flag=1;
end
% compute the new realization
T = paramsValue{1};
R = SS2FWR( inv(T)*Rini.P*T, inv(T)*Rini.Q, Rini.R*T, Rini.S );
\end{verbatim}
and the FWS object is defined by
\begin{verbatim}
Rini =  SS2FWR(A,B,C,D);
S = FWS( Rini, [], @Rfun_SS, [], 'T', eye(R.n));
\end{verbatim}
In that case, the optimization process will have to compute for each iteration a new realization (with the \matlab{Rfun} function)
and then compute the associated FWL measure; whereas in the first case, the FWL measure is directly compute from the $\mt{U}$,
$\mt{Y}$, $\mt{W}$ matrices.
\desc{See also}
\funcName[@FWR/FWR]{FWR}
\end{command}


