\begin{command}[@FWR/implementVHDL]{implementVHDL}
	\desc{Purpose}
Create the associated fixed-point algorithm in VHDL.
Two files are generated \matlab{'xxxx\_entity.vhd'} and \matlab{'xxx\_types.vhd'} where
xxxx is the name given (\matlab{'myFilter'} by default)
	\desc{Syntax}
\matlab{implementVHDL( R,fileName)}
	\desc{Parameters}
		\begin{tabular}{l@{\ :\ }p{9cm}}
\matlab{R} &  FWR object                                      \\
\matlab{fileName} &  name of the function (default=\matlab{myFilter})\\
		\end{tabular}
	\desc{Description}
Generate two VHDL files (named "myFilter\_entity.vhd" and "myFilter\_types.vhd" by default)
that realizes the fixed-point algorithm corresponding to the
realization.\\
All the wordlengths and the fixed-point positions should be first computed by adjusting the FPIS with
\funcName{@FWR/setFPIS}.\\
The files \matlab{@FWR/private/FP\_types.vhd.template} and \matlab{@FWR/private/FP\_entity.vhd.template} are used as a
template.
	\desc{Example}
This function produces a \texttt{myFilter\_types.vhdl} file
\begin{lstlisting}[language=VHDL]
library IEEE;
use IEEE.STD_LOGIC_1164.all;
use IEEE.STD_LOGIC_arith.all;
use IEEE.STD_LOGIC_SIGNED.all;
-- purpose: filtering (generic fixed-point specificatin)
-- type   : sequential/arithmetic
-- inputs : u(n)
-- output : y(n)
-- author : automatically generated by
-- date   : 08-Dec-2008 18:41:00
package FP_types is
-- input data with FP format (16,4,11)
subtype datain is integer range -2**15 to 2**15-1;
-- filtered output data with FP format (16,4,11)
subtype dataout is integer range -2**15 to 2**15-1;
-- states
subtype state1 is integer range -2**15 to 2**15-1;   -- format (16,5,10)
subtype state2 is integer range -2**15 to 2**15-1;   -- format (16,4,11)
subtype state3 is integer range -2**15 to 2**15-1;   -- format (16,3,12)
-- intermediate variables
end FP_types;
\end{lstlisting}
It also produces a \texttt{myFilter\_entity.vhdl} file
\begin{lstlisting}[language=VHDL]
library IEEE;
use IEEE.STD_LOGIC_1164.all;
use IEEE.STD_LOGIC_arith.all;
use IEEE.STD_LOGIC_SIGNED.all;
library work;
use work.FP_types.all;
entity myFilter is
port (
rstb    : in  std_logic; -- asynchronous reset asynchrone active low
clk     : in  std_logic; -- global clock
u       : in  datain; -- input data
y       : out dataout); -- filtered output
end myFilter;
architecture RTL of myFilter is
-- states
signal xn1 : state1 := 0;
signal xn2 : state2 := 0;
signal xn3 : state3 := 0;
-- intermediate variables
begin
-- output(s)
y <= (    xn1 * 22006 + xn2 * 5304 + xn3 * 650 + u   * 1614) / 2**14;
S1: process(rstb,clk)
begin
if rstb = '0' then                  -- asynchronous reset
xn1 <= 0;
xn2 <= 0;
xn3 <= 0;
elsif clk'event and clk = '1' then  -- rising clock edge
-- states
xn1 <= (    xn1 * 18120 + xn2 * (-8813) + xn3 * 239 + u   * 11003) / 2**15;
xn2 <= (    xn1 * 17627 + xn2 * 1591 + xn3 * (-2919) + u   * (-5304)) / 2**14;
xn3 <= (    xn1 * 3824 + xn2 * 23349 + xn3 * (-2387) + u   * 5196) / 2**15;
end if;
end process S1;
end RTL;
\end{lstlisting}
\desc{See also}
\funcName[@FWR/implementLaTeX]{implementLaTeX}, \funcName[@FWR/implementMATLAB]{implementMATLAB}
\end{command}


