\begin{command}[@FWR/implementMATLAB]{implementMATLAB}
	\desc{Purpose}
Create the associated fixed-point algorithm in Matlab language
(it uses integer to simulate fixed-point). The algorithm is written in a file (by default, in \matlab{'myFilter.m'} file)
	\desc{Syntax}
\matlab{implementMATLAB( R,fileName)}
	\desc{Parameters}
		\begin{tabular}{l@{\ :\ }p{9cm}}
\matlab{R} &  FWR object                                                  \\
\matlab{fileName} &  filename of the created function (default=\matlab{'myFilter'})\\
		\end{tabular}
	\desc{Description}
Generate a Matlab file (named \matlab{'myFilter.m'} by default) that emulates the fixed-point algorithm
corresponding to the realization. The rounding operations are realized by the \texttt{floor} function.
All the wordlengths and the fixed-point positions should be first computed by adjusting the FPIS with
\funcName{@FWR/setFPIS}.\\
The file \matlab{@FWR/private/myFilter.m.template} is used as a
template.
	\desc{Example}
It creates a matlab file like
\begin{lstlisting}%[language=matlab]
% Fixed-point algorithm in Matlab language
% (it uses integer to simulate fixed-point)
%
% y = myFilter(u)
%
% y: filtered output(s)
% u: intput(s)
%
% date: 08-Dec-2008 18:12:26
% Automatically generated by implementMATLAB / FWRToolbox
function y = myFilter(u)
% initialize
u = round(2.^11.*u);
y = zeros( size(u,1), 1 );
xn = zeros(3,1);
xnp = zeros(3,1);
for i=1:size(u,1)
% intermediate variables
Acc0 = xn(1) * 18120;
Acc0 = Acc0 + xn(2) * -8813;
Acc0 = Acc0 + xn(3) * 239;
Acc0 = Acc0 + u(i)  * 11003;
xnp(1) = floor( Acc0/2^15 );
Acc1 = xn(1) * 17627;
Acc1 = Acc1 + xn(2) * 1591;
Acc1 = Acc1 + xn(3) * -2919;
Acc1 = Acc1 + u(i)  * -5304;
xnp(2) = floor( Acc1/2^14 );
Acc2 = xn(1) * 3824;
Acc2 = Acc2 + xn(2) * 23349;
Acc2 = Acc2 + xn(3) * -2387;
Acc2 = Acc2 + u(i)  * 5196;
xnp(3) = floor( Acc2/2^15 );
% output(s)
Acc3 = xn(1) * 22006;
Acc3 = Acc3 + xn(2) * 5304;
Acc3 = Acc3 + xn(3) * 650;
Acc3 = Acc3 + u(i)  * 1614;
y(i)   = floor( Acc3/2^14 );
%permutations
xn = xnp;
end
y = 2.^-11.*y;
\end{lstlisting}
It could be used to compute the fixed-point output of the associated realization, with the fixed-point
algorithm.
\begin{verbatim}
>> u=10*rand(1000,1);
>> y=myFilter(u);
\end{verbatim}
\desc{See also}
\funcName[@FWR/implementLaTeX]{implementLaTeX}, \funcName[@FWR/implementVHDL]{implementVHDL}
\end{command}


