\begin{command}[@FWR/implementLaTeX]{implementLaTeX}
	\desc{Purpose}
Return the associated fixed-point algorithm described in \LaTeX
(to be used with package \matlab{algorithm2e})
	\desc{Syntax}
\matlab{code = implementLaTeX( R)         }\\
\matlab{code = implementLaTeX( R, caption)}
	\desc{Parameters}
		\begin{tabular}{l@{\ :\ }p{9cm}}
\matlab{R} &  FWR object                                      \\
\matlab{caption} &  caption used to describe the algorithm    \\
\matlab{} &  (default = \matlab{'Numerical fixed-point algorithm ...'})\\
		\end{tabular}
	\desc{Description}
Return the associated fixed-point algorithm in \LaTeX. It uses the package
\textit{algorithm2e}. All the wordlengths and the fixed-point positions should be
first computed by adjusting the FPIS with
\funcName{@FWR/setFPIS}.\\
The file \matlab{@FWR/private/myFilter.tex.template} is used as a
template.
	\desc{Example}
It returns \LaTeX-code like this
\begin{lstlisting}[language={[LaTeX]tex}]
\begin{algorithm}[h]
\caption{Numerical fixed-point algorithm ...}
\KwIn{$u$: 16 bits integer}
\KwOut{$y$: 16 bits integer}
\KwData{$xn, xnp$: array [1..13] of 16 bits integers}
\KwData{$Acc$: 32 bits integer}
\SetLine
\Begin{
\tcp{\emph{Intermediate variables}}
$Acc \leftarrow (xn(1) * 18120)$\;
$Acc \leftarrow Acc + (xn(2) * -8813)$\;
$Acc \leftarrow Acc + (xn(3) * 239)$\;
$Acc \leftarrow Acc + (u     * 11003)$\;
$xnp(1) \leftarrow Acc >> 15$\;
$Acc \leftarrow (xn(1) * 17627)$\;
$Acc \leftarrow Acc + (xn(2) * 1591)$\;
$Acc \leftarrow Acc + (xn(3) * -2919)$\;
$Acc \leftarrow Acc + (u     * -5304)$\;
$xnp(2) \leftarrow Acc >> 14$\;
$Acc \leftarrow (xn(1) * 3824)$\;
$Acc \leftarrow Acc + (xn(2) * 23349)$\;
$Acc \leftarrow Acc + (xn(3) * -2387)$\;
$Acc \leftarrow Acc + (u     * 5196)$\;
$xnp(3) \leftarrow Acc >> 15$\;
\tcp{\emph{Outputs}}
$Acc \leftarrow (xn(1) * 22006)$\;
$Acc \leftarrow Acc + (xn(2) * 5304)$\;
$Acc \leftarrow Acc + (xn(3) * 650)$\;
$Acc \leftarrow Acc + (u     * 1614)$\;
$y      \leftarrow Acc >> 14$\;
\tcp{\emph{Permutations}}
$xn \leftarrow xnp$\;
}
\end{algorithm}
\end{lstlisting}
That corresponds to the algorithm \ref{algo:implementLaTeX:algo}.
\begin{algorithm}[h]
\caption{Numerical fixed-point algorithm ...\label{algo:implementLaTeX:algo}}
\KwIn{$u$: 16 bits integer}
\KwOut{$y$: 16 bits integer}
\KwData{$xn, xnp$: array [1..13] of 16 bits integers}
\KwData{$Acc$: 32 bits integer}
\SetLine
\Begin{
\tcp{\emph{Intermediate variables}}
$Acc \leftarrow (xn(1) * 18120)$\;
$Acc \leftarrow Acc + (xn(2) * -8813)$\;
$Acc \leftarrow Acc + (xn(3) * 239)$\;
$Acc \leftarrow Acc + (u     * 11003)$\;
$xnp(1) \leftarrow Acc >> 15$\;
$Acc \leftarrow (xn(1) * 17627)$\;
$Acc \leftarrow Acc + (xn(2) * 1591)$\;
$Acc \leftarrow Acc + (xn(3) * -2919)$\;
$Acc \leftarrow Acc + (u     * -5304)$\;
$xnp(2) \leftarrow Acc >> 14$\;
$Acc \leftarrow (xn(1) * 3824)$\;
$Acc \leftarrow Acc + (xn(2) * 23349)$\;
$Acc \leftarrow Acc + (xn(3) * -2387)$\;
$Acc \leftarrow Acc + (u     * 5196)$\;
$xnp(3) \leftarrow Acc >> 15$\;
\tcp{\emph{Outputs}}
$Acc \leftarrow (xn(1) * 22006)$\;
$Acc \leftarrow Acc + (xn(2) * 5304)$\;
$Acc \leftarrow Acc + (xn(3) * 650)$\;
$Acc \leftarrow Acc + (u     * 1614)$\;
$y      \leftarrow Acc >> 14$\;
\tcp{\emph{Permutations}}
$xn \leftarrow xnp$\;
}
\end{algorithm}
\desc{See also}
\funcName[@FWR/algorithmLaTeX]{algorithmLaTeX}, \funcName[@FWR/implementMATLAB]{implementMATLAB}, \funcName[@FWR/implementVHDL]{implementVHDL}
\end{command}


