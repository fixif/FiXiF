\begin{command}[@FWR/algorithmCfloat]{algorithmCfloat}
	\desc{Purpose}
Return the algorithm associated to this realization.
The algorithm is written in C-code with float
	\desc{Syntax}
\matlab{code = algorithmCfloat( R, funcName)}
	\desc{Parameters}
		\begin{tabular}{l@{\ :\ }p{9cm}}
\matlab{code} &  resulting C-code (with float)                  \\
\matlab{R} &  FWR object                                        \\
\matlab{funcName} &  name of the C-function (default=\matlab{myFilter})\\
		\end{tabular}
	\desc{Description}
Transform in \texttt{C}-code with \texttt{float} the algorithm of the realization:
\begin{align*}
&\text{[i]} & JT(k+1) & \leftarrow MX(k) + NU(k)\\
&\text{[ii]} & X(k+1)  & \leftarrow KT(k+1) + PX(k) + QU(k)\\
&\text{[iii]} & Y(k)    & \leftarrow LT(k+1) + RX(k) + SU(k)
\end{align*}
All the operations with matrices are expanded, and null multiplications are removed, identity multiplications are simplified, etc.\\
The input or a pointer to the vector of inputs is given to the function. The function returns the output or a pointer to a vector of putputs.\\
The intermediate variables are stored in a variable \matlab{T}. The states are stored in \matlab{static} variables \matlab{xn} and \matlab{xnp} (\matlab{xnp} is not necessary if $P$ is upper triangular), and a permutation of the vector (a permutation of the pointer to the vector) is performed for the next call.
	\desc{Example}
\begin{lstlisting}[language=C]
>> algorithmCfloat(R)
ans =
float myFilter( float u)
{
// states
static float* xn = (float*) calloc( 8*sizeof(float));
// intermediate variables
float T = -0.6630104844*xn[0] + 2.9240526562*xn[1] + -4.8512758825*xn[2]
+ 3.5897338871*xn[3] + 0.0000312390*xn[4] + 0.0001249559*xn[5]
+ 0.0001874339*xn[6] + 0.0001249559*xn[7] + 0.0000312390*u    ;
// output(s)
y = T    ;
// states
xn[0] = xn[1];
xn[1] = xn[2];
xn[2] = xn[3];
xn[3] = T    ;
xn[4] = xn[5];
xn[5] = xn[6];
xn[6] = xn[7];
xn[7] = u    ;
}
\end{lstlisting}
\desc{See also}
\funcName[@FWR/algorithmLaTeX]{algorithmLaTeX}
\end{command}


