\begin{command}[@FWR/scalprodMATLAB]{scalprodMATLAB}
	\desc{Purpose}
Write the MATLAB code corresponding to a fixed-point scalar product
(the vector of coefficient \matlab{P} by the vector of variables \matlab{name}).
Ex: P(1)*name(1) + P(2)*name(2) + ... + P(n)*name(n)
	\desc{Syntax}
\matlab{scalprodMATLAB( fiel, P, name, gamma, shift, strAcc)}
	\desc{Parameters}
		\begin{tabular}{l@{\ :\ }p{9cm}}
\matlab{file} &  file id, where the scalar product is written  \\
\matlab{P} &  vector of coefficients used in the scalar product\\
\matlab{name} &  name of the variables                         \\
\matlab{gamma} &  fractional part of the coefficients P        \\
\matlab{shift} &  shift to apply after each multiplication     \\
\matlab{strAcc} &  name of the accumulator                     \\
		\end{tabular}
	\desc{Description}
\begin{center}\I{Internal function}\end{center}
This function is called by \funcName[@FWR/implementMATLAB]{implementMATLAB} for each scalar product to be done.\\
It writes the corresponding MATLAB code in a file.\\
\matlab{P} corresponds to the vector of coefficients to use, and \matlab{name} to the vector of variables' name to use.
\desc{See also}
\funcName[@FWR/implementMATLAB]{implementMATLAB}, \funcName[@FWR/scalprodCfloat]{scalprodCfloat}, \funcName[@FWR/scalprodVHDL]{scalprodVHDL}
\end{command}


