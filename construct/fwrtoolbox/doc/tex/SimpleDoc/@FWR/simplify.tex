\begin{command}[@FWR/simplify]{simplify}
	\desc{Purpose}
Simplify (if possible) a FWR, by removing the non-necessary intermediate variables and states
	\desc{Syntax}
\matlab{Rs = simplify( R, level, tol)}
	\desc{Parameters}
		\begin{tabular}{l@{\ :\ }p{9cm}}
\matlab{Rs} &  simplified FWR object                                      \\
\matlab{R} &  FWR to be simplified                                        \\
\matlab{level	} &  level of simplification                                \\
\matlab{} &  0 : simplify only null terms                                 \\
\matlab{} &  1(default) : simplify lines with only one term (substitution)\\
\matlab{} &  2 : simplify lines with 2 terms, etc...                      \\
\matlab{} &  level $>1$ may increase the complexity                       \\
\matlab{tol} &  tolerance (default=1e-14)                                 \\
		\end{tabular}
	\desc{Description}
Due to the sparsity of some realizations (for example, the FFT ones created by \funcName{FFT2FWR}),
some simplifications in the realization can be provided. Null intermediate variables can appear and must be propagated,
and intermediate variables that are sometimes set equal to an other value (without any other computations) must be removed.
Let consider a realization $\mathcal{R}:=(Z,l,m,n,p)$. Computing $T(k+1)$, $X(k+1)$ and $Y(k)$ for each step according
to the associated algorithm is equivalent to compute
\begin{equation}
Z' .\begin{pmatrix} T(k+1)\\ X(k)\\ U(k)\end{pmatrix} \text{\ with\ } Z' \triangleq Z + \begin{pmatrix}I_l \\ &0 \end{pmatrix}
\end{equation}
This function simplifies (if possible) the realization by applying two transformations.\\
The first transform removes the intermediate variables that are null (they could appear for the FFT realization,
because of the real or imaginary parts that are null). It can be
described by the following algorithm:\\
\begin{algorithm}[H]
\caption{\label{algo:null_values}Remove the null values}
\SetLine
%\While{ $\exists i\leq l+n$ such as $Z'_{i,j}=0 \ \forall 1 \leq j \leq l+n+m$}
\While{ it exists $i\leq l+n$ such as $Z'_{i,\bullet}$ is a null vector }
{
\tcp{\I{Remove $i$-th intermediate variable or state}}
remove $i$-th row and $i$-th column of $Z'$\;
decrease $l$ or $n$\;
}
\end{algorithm}
In some cases, various intermediate variables are only equal  to another intermediate variable.\\ For example,
if $T_2 \leftarrow aT_1$ and $T_3 \leftarrow bT_2$, when it is possible to substitute $T_3$ to $T_2$ if $a$ or
$b$ $\in\{-1,1\}$ (in order to preserve the parametrization).\\
The algorithm \ref{algo:substitution} allows the substitution of an intermediate variable by an other one.\\
\begin{algorithm}[H]
\caption{\label{algo:substitution}Substitute the intermediate variables}
\SetLine
%\While{ $\exists i\leq l$ such as $\exists!j$ such as $Z'_{i,k}=\delta_{j,k}Z'_{i,j} \forall 1 \leq k \leq l+n+m$}
\While{ it exists $i \leq l$ such as $Z'_{i,\bullet}$ has only one non-null element $Z'_{i,j}$}
{
\tcp{\I{then we have something like $T_i\leftarrow aT_j$}}
%\If{$\forall l \neq i$, $Z'_{l,i}\neq0 \Rightarrow \pa{Z'_{l,j}=0 \text{\ and\ } \pa{ Z'_{i,j}=\pm1 \text{\ or\ } Z'_{l,i}=\pm1 } }$}
\If{ for all $k \neq i$, $Z'_{k,i}\neq0$ implies $\pa{Z'_{k,j}=0 \text{\ and\ } \pa{ Z'_{i,j}=\pm1 \text{\ or\ } Z'_{k,i}=\pm1 } }$}
{
\tcp{\I{Substitution}}
\For{$1\leq k \leq l+n+p$ such as $Z'_{k,i}\neq0$}
{
$Z'_{k,j} \leftarrow \varepsilon Z'_{k,i}. Z'_{i,j}$\\
with $\varepsilon = \text{sign}(k,i).\text{sign}(i,j).\text{sign}(k,j)$\\
and $\text{sign}(p,q)=\begin{cases}-1 & \text{if\ } p \leq l \text{\ and\ } q \leq l \\ 1 & \text{otherwise}\end{cases}$
}
\tcp{\I{Remove $i$-th intermediate variable}}
remove $i$-th row and $i$-th column of $Z'$\;
decrease $l$\;
}
}
\end{algorithm}
The term $\varepsilon$ in that algorithm is introduced to taking in consideration the $-J$ in the definition of $Z$
(eq. \eqref{eq:def_Z})(this $-J$ was introduced in \cite{Hila07b} in order to simplify the sensitivities and
roundoff measure).
\begin{remark}
It is also possible to consider the substitution when $T_i$ is composed various terms. For example, $T_3
\leftarrow aT_1 + bT_2$ and $T_4 \leftarrow cT_3$ become $T_4 \leftarrow acT_1+bcT_2$ if $c=\pm1$ or
$(a=\pm1 \text{\ and\ }b=\pm1)$. In that case, this transformation can increase the complexity of the computations
(since an intermediate variable that is substituted is used twice or more).
\end{remark}
The input \matlab{level} can set the level of the substitution. \matlab{0} means that only the null terms are removed, and \matlab{1}
only the terms like $T_2 \leftarrow T_3$ are removed. With a greater value, the function considers the substitution if an
intermediate variable is composed from various terms, and \matlab{level} gives the maximum number of terms.
	\desc{Example}
The $FFT_4$ transform first corresponds to the following algorithm (see \funcName{FFT2FWR})
\begin{algorithm}[h]
\caption{$FFT_4$ without any simplification\label{algo:FFT4_1}}
\KwIn{$u$: array [1..4] of reals}
\KwOut{$y$: array [1..8] of reals}
\KwData{$T$: array [1..16] of reals}
\SetLine
\Begin{
\tcp{\emph{Intermediate variables}}
$T_{ 1} \leftarrow u(1)   + u(3)  $\;
$T_{ 2} \leftarrow 0$\;
$T_{ 3} \leftarrow u(1)   + -u(3)  $\;
$T_{ 4} \leftarrow 0$\;
$T_{ 5} \leftarrow u(2)   + u(4)  $\;
$T_{ 6} \leftarrow 0$\;
$T_{ 7} \leftarrow u(2)   + -u(4)  $\;
$T_{ 8} \leftarrow 0$\;
$T_{ 9} \leftarrow T_{ 1}$\;
$T_{10} \leftarrow T_{ 2}$\;
$T_{11} \leftarrow T_{ 3}$\;
$T_{12} \leftarrow T_{ 4}$\;
$T_{13} \leftarrow T_{ 5}$\;
$T_{14} \leftarrow T_{ 6}$\;
$T_{15} \leftarrow T_{ 8}$\;
$T_{16} \leftarrow -T_{ 7}$\;
\tcp{\emph{Outputs}}
$y(1)   \leftarrow T_{ 9} + T_{13}$\;
$y(2)   \leftarrow T_{10} + T_{14}$\;
$y(3)   \leftarrow T_{11} + T_{15}$\;
$y(4)   \leftarrow T_{12} + T_{16}$\;
$y(5)   \leftarrow T_{ 9} + -T_{13}$\;
$y(6)   \leftarrow T_{10} + -T_{14}$\;
$y(7)   \leftarrow T_{11} + -T_{15}$\;
$y(8)   \leftarrow T_{12} + -T_{16}$\;
}
\end{algorithm}
When only the null terms are removed (with \matlab{level}=0), the algorithm becomes:
\begin{algorithm}[h]
\caption{$FFT_4$ with null terms removed\label{algo:FFT4_2}}
\KwIn{$u$: array [1..4] of reals}
\KwOut{$y$: array [1..8] of reals}
\KwData{$T$: array [1..8] of reals}
\SetLine
\Begin{
\tcp{\emph{Intermediate variables}}
$T_{1} \leftarrow u(1)  + u(3) $\;
$T_{2} \leftarrow u(1)  + -u(3) $\;
$T_{3} \leftarrow u(2)  + u(4) $\;
$T_{4} \leftarrow u(2)  + -u(4) $\;
$T_{5} \leftarrow T_{1}$\;
$T_{6} \leftarrow T_{2}$\;
$T_{7} \leftarrow T_{3}$\;
$T_{8} \leftarrow -T_{4}$\;
\tcp{\emph{Outputs}}
$y(1)  \leftarrow T_{5} + T_{7}$\;
$y(2)  \leftarrow 0$\;
$y(3)  \leftarrow T_{6}$\;
$y(4)  \leftarrow T_{8}$\;
$y(5)  \leftarrow T_{5} + -T_{7}$\;
$y(6)  \leftarrow 0$\;
$y(7)  \leftarrow T_{6}$\;
$y(8)  \leftarrow -T_{8}$\;
}
\end{algorithm}
Then, with a complete substitution (\matlab{level}=1), the final algorithm is:
\begin{algorithm}[h]
\caption{$FFT_4$ with substitutions (1 term)\label{algo:FFT4_3}}
\KwIn{$u$: array [1..4] of reals}
\KwOut{$y$: array [1..8] of reals}
\KwData{$T$: array [1..5] of reals}
\SetLine
\Begin{
\tcp{\emph{Intermediate variables}}
$T_{1} \leftarrow u(1)  + u(3) $\;
$T_{2} \leftarrow u(1)  + -u(3) $\;
$T_{3} \leftarrow u(2)  + u(4) $\;
$T_{4} \leftarrow u(2)  + -u(4) $\;
\tcp{\emph{Outputs}}
$y(1)  \leftarrow T_{1} + T_{3}$\;
$y(2)  \leftarrow 0$\;
$y(3)  \leftarrow T_{2}$\;
$y(4)  \leftarrow -T_{4}$\;
$y(5)  \leftarrow T_{1} + -T_{3}$\;
$y(6)  \leftarrow 0$\;
$y(7)  \leftarrow T_{2}$\;
$y(8)  \leftarrow T_{4}$\;
}
\end{algorithm}
\desc{See also}
\funcName[FFT2FWR]{FFT2FWR}
\end{command}


