\begin{command}[@FWR/subsref]{subsref}
	\desc{Purpose}
Subscripted reference for FWR object
here, R.prop is equivalent to get(R,\matlab{prop})
	\desc{Syntax}
\matlab{value = subsref(R,Sub)}
	\desc{Parameters}
		\begin{tabular}{l@{\ :\ }p{9cm}}
\matlab{value} &  returned value        \\
\matlab{R} &  FWR object                \\
\matlab{Sub} &  layers of subreferencing\\
		\end{tabular}
	\desc{Description}
These functions are called internally when operators \matlab{[]}, \matlab{()}
and \matlab{.} are applied on a FWR object.\\
Only the operator \matlab{.} is valid, and links to \funcName[@FWR/set]{set} and \funcName[@FWR/get]{get}
functions. The command \matlab{R.field} returns the field \matlab{field} of \matlab{R}
(internally, \matlab{get(R,'field')} is called), and \matlab{R.field=vaue} set the field
\matlab{field} of \matlab{R}
	\desc{Example}
\matlab{R.P .* R.WP}\\
\matlab{R.P(1,:)}\\
\matlab{R.Z(3,3) = 0;}\\
\matlab{R.WZ = zeros( size(R.WZ) );}
\desc{See also}
\funcName[@FWR/set]{set}, \funcName[@FWR/get]{get}, \funcName[@FWR/subsasgn]{subsasgn}, \funcName[@FWS/subsref]{subsref}
\end{command}


