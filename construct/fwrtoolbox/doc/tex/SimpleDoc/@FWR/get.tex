\begin{command}[@FWR/get]{get}
	\desc{Purpose}
Get some properties of a FWR object
(or list the properties if \matlab{propName} is ignored)
	\desc{Syntax}
\matlab{value = get(R, propName)}
	\desc{Parameters}
		\begin{tabular}{l@{\ :\ }p{9cm}}
\matlab{value} &  value of the property            \\
\matlab{R} &  FWR object                           \\
\matlab{propName } &  name of the property (string)\\
		\end{tabular}
	\desc{Description}
This function is most of the time called by \funcName{@FWR/subsref}.\\
The value of every field (\matlab{l}, \matlab{m}, \matlab{n} and \matlab{p} ; \matlab{J},
\matlab{K}, \matlab{L}, \matlab{M}, \matlab{N}, \matlab{P}, \matlab{Q}, \matlab{R} and \matlab{S};
\matlab{Z} ; \matlab{WJ}, \matlab{WK}, \matlab{WL}, \matlab{WM}, \matlab{WN}, \matlab{WP},
\matlab{WQ}, \matlab{WR} and \matlab{WS} ;	\matlab{WZ} ; \matlab{AZ}, \matlab{BZ}, \matlab{CZ}
and \matlab{AZ}) can be evaluated with this command, but \matlab{l}, \matlab{m}, \matlab{n}, \matlab{p},
\matlab{AZ}, \matlab{BZ}, \matlab{CZ} and \matlab{AZ} cannot be modified.\\
Changing \matlab{Z} changes fields \matlab{J} to \matlab{S}, and reciprocally (this is
the same with \matlab{WZ}).
\desc{See also}
\funcName[@FWR/set]{set}, \funcName[@FWR/subsref]{subsref}, \funcName[@FWR/subsasgn]{subsasgn}, \funcName[@FWS/get]{get}
\end{command}


