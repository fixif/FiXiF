\begin{command}[@FWR/FWR]{FWR}
	\desc{Purpose}
FWR class' constructor
	\desc{Syntax}
\matlab{R = FWR()                            }\\
\matlab{R = FWR(R1)                          }\\
\matlab{R = FWR(J,K,L,M,N,P,Q,R,S, fp, block)}
	\desc{Parameters}
		\begin{tabular}{l@{\ :\ }p{9cm}}
\matlab{R} &  FWR object created                                                                                                                                     \\
\matlab{R1} &  FWR object to be copied                                                                                                                               \\
\matlab{J,K,...,S} &  matrices of the realization                                                                                                                    \\
\matlab{fp} &  fixed-point or floating-point representation                                                                                                          \\
\matlab{} & \matlab{'fixed'} (default value) or \matlab{'floating'}                                                                                                                    \\
\matlab{block} &  block-representation scheme. The coefficients in a same block share the same representation (same scale factor, etc...). Take the following values:\\
\matlab{} &  \matlab{'full'}: same representation for all coefficients of R                                                                                                   \\
\matlab{} &  \matlab{'natural'} (default value): blocks are made of matrices J,K,L,M,N,P,Q,R,S                                                                                \\
\matlab{} &  \matlab{'none'}: each coefficient has its own representation (according to its value)                                                                            \\
		\end{tabular}
	\desc{Description}
Constructor of the \matlab{FWR} class.\\
It could create an empty object ($l=m=n=p=0$), copy an object or create a \matlab{FWR} object from matrices $J$ to $S$: in that case, the other parameters are deduced (only $0$, $1$ or $-1$ are considered as exactly implemented).
\desc{See also}
\funcName[@FWS/FWS]{FWS}
\end{command}


