\begin{command}[@FWR/RNGcl]{RNG\_cl}
	\desc{Purpose}
Compute the closed-loop Roundoff Noise Gain
	\desc{Syntax}
\matlab{G = RNG( R, Sysp)          }\\
\matlab{[G, dZ] = RNG(R, Sysp, tol)}
	\desc{Parameters}
		\begin{tabular}{l@{\ :\ }p{9cm}}
\matlab{G} &  roundoff noise gain                                 \\
\matlab{dZ} &  number of non-trivial parameters (used by @FWS/RNG)\\
\matlab{R} &  FWR object                                          \\
\matlab{Sysp} &  plant (to be controlled)                         \\
\matlab{tol} &  tolerance on trivial parameters (default=1e-8)    \\
		\end{tabular}
	\desc{Description}
This function computes the Roundoff Noise Gain (in closed-loop context).\\
The Roundoff Noise Gain is the output noise power computed in a specific computational scheme : the noises are supposed to
appear only after each multiplication and are modeled by centered white noise statistically independent.\\ Each noise is
supposed to have the same power $\sigma_0^2$ (determined by the wordlength choosen for all the variables and coefficients).\\
The Roundoff Noise Gain is defined by
\begin{equation}
\bar{G} \triangleq \frac{\bar{P}}{\sigma_0^2}
\end{equation}
where $\bar{P}$ is the output roundoff noise power (the global noise added on the output of the plant).
It could be computed by
\begin{equation}
\bar{G} = tr\pa{ d_Z (\bar{M}_2^\top \bar{M}_2 + \bar{M}_1^\top \bar{W}_o \bar{M}_1) }
\end{equation}
with
\begin{eqnarray}
\bar{M}_1 &=& \begin{pmatrix}
B_2LJ^{-1} & 0 & B_2 \\
KJ^{-1} & I_n & 0
\end{pmatrix},  \\
\bar{M}_2 &=& \begin{pmatrix} D_{12}LJ^{-1} & 0 & D_{12} \end{pmatrix}
\end{eqnarray}
and the matrix $d_Z$ is a diagonal matrix defined by
\begin{equation}
\pa{d_Z}_{i,i} \triangleq \text{number of non-trivial parameters in the i\textsuperscript{th} row of $Z$}
\end{equation}
The trivial parameters considered are $0$, $1$ and $-1$ because they do not imply a multiplication.
\desc{See also}
\funcName[@FWR/RNG]{RNG}, \funcName[@FWS/RNGcl]{RNG\_cl}
	\desc{References}
\cite{Hila08b} T.�Hilaire, P.�Chevrel, and J.�Whidborne. Finite wordlength controller realizations using the specialized implicit form. Technical Report RR-6759, INRIA, 2008.
\end{command}


