\begin{command}[@FWR/computationalCost]{computationalCost}
	\desc{Purpose}
Give the number of additions and multiplications implied in the realization
	\desc{Syntax}
\matlab{[add, mul] = computationalCost( R, tol)}
	\desc{Parameters}
		\begin{tabular}{l@{\ :\ }p{9cm}}
\matlab{add} &  number of additions             \\
\matlab{mul} &  number of multiplications       \\
\matlab{R} &  FWR object                        \\
\matlab{tol} &  tolerance (default value = 1e-8)\\
		\end{tabular}
	\desc{Description}
The number of additions and multiplications is based on the number of trivial parameters
and null parameters.\\
The evaluation is based on the following proposition, applied on the three steps [i],
[ii] and [iii] of algorithm \eqref{eq:def_implicit} :
\begin{proposition}
Let $Y\in\Rbb{a}{b}$ be a constant, and $V\in\Rbb{b}{1}$ a variable.\\
The calculus $YV$ needs $a(b-1)-n^0_Y$ additions and $ab-n^1_Y$ multiplications, where
$n^0_Y$ is the number of null elements of $Y$ and $n^1_Y$ is the number of trivial elements
($0$,$1$,$-1$) of $Y$ (these elements don't imply a multiplication)
\end{proposition}
Then, the algorithm requires $(l+n+p)(l+m+n-1)-l-n^0_Z$ additions and $(l+n+p)(l+m+n)-n^1_Z$
multiplications.
\end{command}


