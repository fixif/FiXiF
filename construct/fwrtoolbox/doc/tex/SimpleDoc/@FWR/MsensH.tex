\begin{command}[@FWR/MsensH]{MsensH}
	\desc{Purpose}
Compute the open-loop transfer function sensitivity measure (and the
sensitivity matrix)
	\desc{Syntax}
\matlab{[M MZ]  = MsensH(R)}
	\desc{Parameters}
		\begin{tabular}{l@{\ :\ }p{9cm}}
\matlab{M} &  sensitivity measure\\
\matlab{MZ} &  sensitivity matrix\\
\matlab{R} &  FWR object         \\
		\end{tabular}
	\desc{Description}
The open-loop transfer function sensitivity measure is defined by
\begin{equation}
M_{L_{2}}^W = \norm{\dede{H}{Z} \times r_{Z}}_{F}^2.
\end{equation}
where $\dede{H}{Z}\in\Rbb{l+n+p}{l+n+q}$ is the \I{transfer function sensitivity matrix}.
It is the matrix of the $L_{2}$-norm of the sensitivity of the transfer function $H$ with
respect to each coefficient  $Z_{i,j}$. It is defined by
\begin{equation}
\pa{\dede{H}{Z}}_{i,j} \triangleq \norm{\dd{H}{Z_{i,j}}}_{2},
\end{equation}
In SISO case, the $M_{L_{2}}^W$ measure is equal to
\begin{equation}
M_{L_{2}}^W = \norm{\dd{H}{Z} \times r_{Z}}_2^2
\end{equation}
and is then an extension to the SIF of the classical state-space sensitivity measure
\begin{equation}
M_{L_2} \triangleq \norm{\dd{H}{A}}_2^2 + \norm{\dd{H}{B}}_2^2 + \norm{\dd{H}{C}}_2^2.
\end{equation}
The $M_{L_2}^W$ measure can be evaluated by the following propositions
\begin{proposition}
\begin{equation}
\dd{H}{Z} =  H_1 \cd H_2
\end{equation}
where $H_1$ and $H_2$ are defined by
\begin{eqnarray}
H_1 : z &\mapsto& C_Z (zI_n-A_Z)^{-1} M_1 + M_2 \\
H_2 : z &\mapsto& N_2 + N_1 (zI_n-A_Z)^{-1} B_Z
\end{eqnarray}
with
\begin{align}
M_1 &\triangleq  \begin{pmatrix} KJ^{-1} & I_n & 0 \end{pmatrix}, &
M_2 &\triangleq  \begin{pmatrix} LJ^{-1} & 0 & I_{p_2} \end{pmatrix}, \\
N_1 &\triangleq  \begin{pmatrix} J^{-1}M \\ I_n \\ 0 \end{pmatrix}, &
N_2 &\triangleq  \begin{pmatrix} J^{-1}N \\ 0 \\ I_{m_2} \end{pmatrix}.
\end{align}
\end{proposition}
\begin{proposition}
The transfer function sensitivity matrix $\dede{H}{Z}$ can be computed as
\begin{equation}
\pa{ \dede{H}{Z} }_{i,j} = \norm{ H_1 E_{i,j} H_2 }_2
\end{equation}
with
\begin{equation}
H_1 E_{i,j} H_2 :=
\left(\begin{array}{cc|c}
A_Z & 0 & B_Z \\
M_1 E_{i,j} N_1 & A & M_1 E_{i,j} N_2 \\
\hline \vspace{-3.5mm}\\
M_2 E_{i,j} N_1 & C_Z & M_2 E_{i,j} N_2
\end{array}\right)
\end{equation}
and $E_{i,j}$ is the matrix of appropriate size with all elements being $0$ except the $(i,j)$th element which is unity.
\end{proposition}
\begin{remark}
In the SISO case, the problem becomes simpler by noting that
\begin{align}
\pa{ \dede{H}{Z} }_{i,j} &=  \norm{ (H_2H_1)_{i,j} }_2 \\
&= \norm{
\left(\begin{array}{cc|c}
A_Z & 0 & B_Z \\
M_1 N_1 & A_Z & M_1 N_2 \\
\hline \vspace{-3.5mm}\\
M_2 N_1 & C_Z & M_2 N_2
\end{array}\right)_{i,j}}_2
\end{align}
The $(l+n+1)\times(l+n+1)$ $H_2$-norm evaluations here require  only $l+n+1$ Lyapunov equations to be solved .
\end{remark}
\desc{See also}
\funcName[@FWR/MsensHcl]{MsensH\_cl}, \funcName[@FWR/wprodnorm]{w\_prod\_norm}
	\desc{References}
\cite{Hila07d}	T.�Hilaire and P.�Chevrel. On the compact
formulation of thederivation of a transfer matrix with respect to
another matrix. Technical Report RR-6760, INRIA, 2008.\\
\cite{Hila06a} T.�Hilaire, P.�Chevrel, and J.-P. Clauzel. Low
parametric sensitivity realization design for FWL implementation
of MIMO controllers : Theory and application to the active control of vehicle longitudinal oscillations. In Proc. of Control Applications of Optimisation CAO'O6, April 2006.\\
\cite{Hila07b} T.�Hilaire, P.�Chevrel, and J.�Whidborne. A
unifying framework for finite wordlength realizations. IEEE Trans.
on Circuits and Systems, 8(54), August 2007.\\
\end{command}


