% Generated by SimpleDoc (�T. Hilaire) - 09-Mar-2009

\begin{longtable}{|p{3.4cm}|p{8.4cm}|}
	\hline \funcName[@FWR/algorithmCfloat]{algorithmCfloat} & Return the algorithm associated to this realization.\\
	\hline \funcName[@FWR/algorithmLaTeX]{algorithmLaTeX} & Return the pseudocode algorithm described in \LaTeX\\
	\hline \funcName[@FWR/computationalCost]{computationalCost} & Give the number of additions and multiplications implied in the realization\\
	\hline \funcName[@FWR/computeW]{computeW} & Compute (or update) the weighting matrices ($W_J$ to $W_S$, and $W_Z$) of a FWR object\\
	\hline \funcName[@FWR/display]{display} & Display the realization (dimensions and Z)\\
	\hline \funcName[@FWR/double]{double} & Convert FWR object to double (return Z matrix)\\
	\hline \funcName[@FWR/FWR]{FWR} & FWR class' constructor\\
	\hline \funcName[@FWR/FWRmat2LaTeX]{FWRmat2LaTeX} & Display a matrix ($Z$ or a sensitivity matrix) in \LaTeX (with \matlab{pmat} \\
	\hline \funcName[@FWR/get]{get} & Get some properties of a FWR object              \\
	\hline \funcName[@FWR/implementLaTeX]{implementLaTeX} & Return the associated fixed-point algorithm described in \LaTeX\\
	\hline \funcName[@FWR/implementMATLAB]{implementMATLAB} & Create the associated fixed-point algorithm in Matlab language                                                  \\
	\hline \funcName[@FWR/implementVHDL]{implementVHDL} & Create the associated fixed-point algorithm in VHDL.               \\
	\hline \funcName[@FWR/l2scaling]{l2scaling} & Perform a $L_2$-scaling on the FWR\\
	\hline \funcName[@FWR/MsensH]{MsensH} & Compute the open-loop transfer function sensitivity measure (and the\\
	\hline \funcName[@FWR/MsensHcl]{MsensH\_cl} & Compute the closed-loop transfer function sensitivity measure (and the\\
	\hline \funcName[@FWR/MsensPole]{MsensPole} & Compute the open-loop pole sensitivity measure    \\
	\hline \funcName[@FWR/MsensPolecl]{MsensPole\_cl} & Compute the closed-loop pole sensitivity measure  \\
	\hline \funcName[@FWR/Mstability]{Mstability} & Compute the closed-loop pole sensitivity stability related measure\\
	\hline \funcName[@FWR/mtimes]{mtimes} & Multiply two FWR (put them in cascade)\\
	\hline \funcName[@FWR/ONP]{ONP} & Compute the Output Noise Power for a FWR object with Roundoff Before\\
	\hline \funcName[@FWR/plus]{plus} & add two FWR object (put them in parallel)\\
	\hline \funcName[@FWR/quantized]{quantized} & Return the quantized realization, according to a fixed-point implementation scheme\\
	\hline \funcName[@FWR/realize]{realize} & Numerically compute the outputs, states and intermediate variables with a given input U.\\
	\hline \funcName[@FWR/relaxedl2scaling]{relaxedl2scaling} & Perform a relaxed-$L_2$-scaling on the FWR.               \\
	\hline \funcName[@FWR/RNG]{RNG} & Compute the open-loop Roundoff Noise Gain\\
	\hline \funcName[@FWR/RNGcl]{RNG\_cl} & Compute the closed-loop Roundoff Noise Gain\\
	\hline \funcName[@FWR/set]{set} & Set some properties of a FWR object\\
	\hline \funcName[@FWR/setFPIS]{setFPIS} & Set the Fixed-Point Implementation Scheme (FPIS) of an FWR object                                               \\
	\hline \funcName[@FWR/sigmatf]{sigma\_tf} & Compute the open-loop transfer function error $\sigma_{\Delta H}^2$)\\
	\hline \funcName[@FWR/simplify]{simplify} & Simplify (if possible) a FWR, by removing the non-necessary intermediate variables and states\\
	\hline \funcName[@FWR/size]{size} & Return the size of the FW Realization\\
	\hline \funcName[@FWR/ss]{ss} & Convert a FWR object into a ss (state-space) object (equivalent state-space)\\
	\hline \funcName[@FWR/subsasgn]{subsasgn} & Subscripted assign for FWR object                          \\
	\hline \funcName[@FWR/subsref]{subsref} & Subscripted reference for FWR object       \\
	\hline \funcName[@FWR/tf]{tf} & Convert a FWR object into a tf object (transfer function)\\
	\hline \funcName[@FWR/TradeOffMeasurecl]{TradeOffMeasure\_cl} & Compute a (pseudo) tradeoff closed-loop measure\\
	\hline \funcName[@FWR/transform]{transform} & Perform a UYW-transformation (similarity on Z)\\
	\hline
\end{longtable}
