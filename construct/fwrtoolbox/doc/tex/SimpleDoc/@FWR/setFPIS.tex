\begin{command}[@FWR/setFPIS]{setFPIS}
	\desc{Purpose}
Set the Fixed-Point Implementation Scheme (FPIS) of an FWR object
(the wordlength may be matrices or scalar. The scalar case is used to set all the wordlength to the same length)
	\desc{Syntax}
\matlab{R = setFPIS( R, betaU, Umax, betaZ, betaT, betaX, betaY, betaADD, betaG, method )}\\
\matlab{R = setFPIS( R, FPIS)                                                            }\\
\matlab{R = setFPIS( R, FPISname,Umax)                                                   }
	\desc{Parameters}
		\begin{tabular}{l@{\ :\ }p{9cm}}
\matlab{R} &  FWR object                                                                                                                         \\
\matlab{FPIS} &  an other Fixed-Point Implementation Scheme (a structure with betaU, Umax, betaZ, betaT, betaX, betaY, betaADD, betaG and method)\\
\matlab{FPISname } &  \matlab{'DSP8'} or \matlab{'DSP16'}                                                                                                          \\
\matlab{betaU} &  wordlength of U (inputs)                                                                                                       \\
\matlab{Umax} &  maximum value of U (necessary to set $\gamma_U$)                                                                                \\
\matlab{betaZ} &  wordlength of the coefficients                                                                                                 \\
\matlab{betaT, betaX, betaY} &  wordlength of the intermediate variables T, the states X and the outputs                                         \\
\matlab{betaADD} &  wordlength of the accumulators                                                                                               \\
\matlab{betaG} &  nb of guard bits in the accumulators                                                                                           \\
\matlab{method} &  \matlab{'RBM'} (default) Roundoff Before Multiplication                                                                                \\
\matlab{} &  \matlab{'RAM'} Roundoff After Multiplication                                                                                                 \\
		\end{tabular}
	\desc{Description}
This function sets the Fixed-Point Implementation Scheme (FPIS).
This structure is composed by:
\begin{itemize}
\item the fixed-point format of the input $(\beta_U,\gamma_U)$
and its maximum magnitude value $\overset{\max}{U}$
\item the fixed-point format of the intermediate variables $(\beta_T,\gamma_T)$
\item the fixed-point format of the states $(\beta_X,\gamma_X)$
\item the fixed-point format of the output $(\beta_Y,\gamma_Y)$
\item the fixed-point format of the coefficients $(\beta_Z,\gamma_Z)$
\item the fixed-point format of the accumulator $(\beta_{ADD}+\beta_{G},\gamma_{ADD})$ ($\beta_G$ guard bits)
\item the right-shift bits after each scalar product $d_{ADD}$ (\matlab{shiftADD})
\item the right-shift bits after each multiplication by a coefficient $d_Z$ (\matlab{shiftZ})
\item the computational scheme : \I{Roundoff After Multiplication} (RAM) or \I{Roundoff Before Multiplication} (RBM)
\end{itemize}
The algorithm
\begin{align*}
&\text{[i]} & JT(k+1) & \leftarrow MX(k) + NU(k)\\
&\text{[ii]} & X(k+1)  & \leftarrow KT(k+1) + PX(k) + QU(k)\\
&\text{[iii]} & Y(k)    & \leftarrow LT(k+1) + RX(k) + SU(k)
\end{align*}
requires to implement $l+n+p$ scalar products.\\
Each scalar product
\begin{equation}
S = \sum_{i=1}^n P_i E_i
\end{equation}
where $\pa{P_i}_{1 \leq i \leq n}$ are given coefficients and
$\pa{E_i}_{1 \leq i \leq n}$ some bounded variables, can be
implemented according to the algorithms \ref{algo:setFPIS:setFPIS:RAM} and
\ref{algo:setFPIS:setFPIS:RBM}, and where $P'_i$, $E'_i$ and $S'_i$ are the integer representation
(according to their fixed-point format) to $P_i$,$E_i$ and
$S_i$.\\
\begin{multicols}{2}{
\begin{algorithm}[H]
\caption{\I{Roundoff After Multiplication} (RAM)\label{algo:setFPIS:setFPIS:RAM}}
$Add\leftarrow 0$\\
\For{$i$ from 0 to $n$}{$Add\leftarrow \pa{ P_i' * E_i' } >> d_i$}
$S'_i \leftarrow Add >> d'_i$
\end{algorithm} \ \\
\begin{algorithm}[H]
\caption{\I{Roundoff Before Multiplication} (RBM)\label{algo:setFPIS:setFPIS:RBM}}
$Add\leftarrow 0$\\
\For{$i$ from 0 to $n$}{$Add\leftarrow \pa{ P_i' >> d_i } * E_i'$}
$S'_i \leftarrow Add >> d_i'$
\end{algorithm}}
\end{multicols}
Of course, $d_i$ represents the right-shift after each multiplication and $d'_i$ represents the final shift. They respectively correspond to the $d_Z$ and $d_{ADD}$ shift in the SIF algorithm.
The user may specify all the wordlengths ($\beta_U$, $\beta_T$, $\beta_X$, $\beta_Y$, $\beta_{ADD}$, $\beta_g$
and $\beta_Z$) and $\overset{\max}{U}$. The binary-point positions are deduced by:
\begin{equation}
\gamma_U = \beta_U - 2 - \floor{ \log_2 \overset{\max}{U} }
\end{equation}
\begin{equation}
\begin{pmatrix} \gamma_T \\ \gamma_X \\ \gamma_Y \end{pmatrix} =
\begin{pmatrix} \beta_T \\ \beta_X \\ \beta_Y \end{pmatrix} - 2.\VecOne{l+n+p}{1} -
\floor{ \log_2 \pa{ \norm{H_{\max}}_{l_1} \overset{\max}{\abs{U}} } }
%		\left\lceil \log_2 \begin{pmatrix} \overset{\max}{\abs{T}} \\ \overset{\max}{\abs{X}} \\ \overset{\max}{\abs{Y}} \end{pmatrix} \right\rceil
\end{equation}
where $\VecOne{k}{l}$ represents the matrix of $\Rbb{k}{l}$ with all coefficients set to 1,
$\norm{.}_{l_1}$ the $l_1$-norm and
\begin{equation}
H_{\max} : z \to N_1 \pa{ zI_n-A_Z }^{-1} B_Z + N_2,
\end{equation}
\begin{equation}
N_1 \triangleq \begin{pmatrix}J^{-1}M \\ I_n \\ C_Z \end{pmatrix}, N_2 \triangleq \begin{pmatrix} J^{-1}N \\ 0 \\ D_Z \end{pmatrix}
\end{equation}
The binary point position\footnote{$\pa{\gamma_Z}_{i,j}$ could be $-\infty$ for null coefficients, but it is not a problem because such coefficients are not implemented} $\gamma_Z$ of the coefficients $Z$ are given by:
\begin{equation}\label{eq:gammaZ}
\gamma_Z = \beta_Z - 2.\VecOne{l+n+p}{l+n+m} - \left \lfloor \strut \log_2 \abs{Z} \right\rfloor
\end{equation}
The fixed-point formats of the additions are given by:
\begin{equation}
\gamma_{ADD} = \beta_{ADD} - \underset{row}{\max} \pa{ \begin{pmatrix} \beta_T \\ \beta_X \\ \beta_Y \end{pmatrix} - \beta_g - \begin{pmatrix} \gamma_T \\ \gamma_X \\ \gamma_Y \end{pmatrix} , \alpha }
\end{equation}
where
\begin{equation}
\alpha = \underset{row}{\max} \pa { \beta_Z-\gamma_Z + \VecOne{l+n+p}{1} \pa{ \begin{pmatrix} \beta_T \\ \beta_X \\ \beta_U \end{pmatrix} - \begin{pmatrix} \gamma_T \\ \gamma_X \\ \gamma_U \end{pmatrix} }^{\hspace{-2mm}\top} }
\end{equation}
and $\underset{row}{\max}(M)$ returns a column vector with the maximum value of each row of $M$.\\
The final alignments are right shifts of $d_{ADD}$ bits, with:
\begin{equation}
d_{ADD} = \gamma_{ADD} - \begin{pmatrix} \gamma_T \\ \gamma_X \\ \gamma_Y \end{pmatrix}
\end{equation}
Denote $\tilde\gamma_Z$ the final binary point position of the coefficients $Z$, according to \I{RAM} or \I{RBM} scheme, and $d_Z$ the shifts needed after each multiplication ($\pa{d_Z}_{i,j}$ is the right shift needed after the multiplication by $Z_{i,j}$) in order to align the format after each multiplication. Then:
\begin{equation}
\tilde\gamma_Z = \begin{cases}
\gamma_Z & \text{if \I{RAM}}\\
\gamma_{ADD}.\VecOne{1}{l+n+m} - \VecOne{l+n+p}{1}. \begin{pmatrix} \gamma_T \\ \gamma_X \\ \gamma_U \end{pmatrix}^{\hspace{-2mm}\top} & \text{if \I{RBM}}
\end{cases}
\end{equation}
and
\begin{equation}
d_Z = \tilde\gamma_Z + \VecOne{l+n+p}{1}. \begin{pmatrix} \gamma_T \\ \gamma_X \\ \gamma_U \end{pmatrix}^{\hspace{-2mm}\top} - \gamma_{ADD}.\VecOne{1}{l+n+m}
\end{equation}
($d_Z$ is a null matrix in \I{RBM} case).\\
With $d_Z$, $\tilde\gamma_Z$, $\gamma_{ADD}$, $d_{ADD}$, $\gamma_T$, $\gamma_X$ and $\gamma_Y$, the fixed-point implementation of the controller is entirely defined.
\desc{See also}
\funcName[@FWR/quantized]{quantized}, \funcName[@FWS/setFPIS]{setFPIS}
	\desc{References}
\cite{Hila08c} T.�Hilaire, D.�M�nard, and O.�Sentieys. Bit accurate roundoff noise analysis of fixed-point linear controllers. In Proc. IEEE International Symposium on Computer-Aided Control System Design (CACSD'08), September 2008.
\end{command}


