\chapter{FiPoGen}

Re-Modélisation d'une partie de FiPoGen 


\smallskip
[ Avantages:
\begin{itemize}
\item Cela organisera la compréhension du système et facilite l'intégration d'un nouveau développeur.
\item Permet une exploration d'autres alternatives
\item facilite le développement et la maintenance du système
\item facilite la réutilisation ]
\end{itemize}

\section{Présentation de FiPoGen}
FiPoGen est un outil développé dans le cadre de la thèse de Benoit Lopez et du projet ANR DEFIS (Design of fixed­-point  embedded  systems,  2011-­2014)  permettant  de  transformer  un  algorithme  en  un graphe d’opérations virgule fixe, avec calcul des erreurs commises, optimisations des largeurs des opérateurs  (sous  contrainte  d’erreur  de  sortie). Dans sa version actuelle, FiPoGen  est  principalement  dédié  à  l'implantation  des  filtres linéaires récursifs.

\section{Les problèmes Liés à la précision Finie} [FWRUserGuide.pdf][Thèse Benoit]
Les filtres numériques sont toujours implémentés dans une précision finie car les calculateurs ont un nombre de bits limités pour représenter les nombres et faire les calculs. \\
Cela introduit 2 effets :
\begin{itemize}
\item le bruit d'arrondi sur les variables - round-off noise
\item la dégradation de performance due à l'arrondi des coefficients - coefficients sensitivity
\end{itemize}
Il est donc nécessaire de s'assurer que ces effets n'entraine pas une dégradation considérable sur la performance du filtre.\\
Ces effets dépendent du:
\begin{itemize}
\item format d'arithmétique choisi (FIP ou FxP)
\item la longueur de mots pour représenter les nombres
\item la réalisation choisie ( structure du filtre )
\end{itemize}



\section{L'architecture de FiPoGen}