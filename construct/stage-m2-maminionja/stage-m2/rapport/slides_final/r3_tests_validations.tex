%==============================================================================
\section{Tests et Validations}
%==============================================================================
\hspace*{2cm}
\frame{\tableofcontents[currentsection]}
%________________________________________
\begin{frame} \FT{Validation du module SLX2SIF }
	\begin{itemize}
	\item Validation sur les 2 types de structures de d'adaptateur pour LWDF
	\begin{center}
	\FIGW{0.7}{adapteur_sif_.png}
	\end{center}
	\item Validation sur un exemple de LWDF de 5è ordre (EUSIPCO15)
	\begin{center}
	\FIGW{0.7}{lwdf_butter5L.png}
	\end{center}
	\end{itemize}
	\textbf{LWDF} : Lattice Wave Digital Filter
\end{frame}

\begin{frame} \FT{Validation du module SLX2SIF}
Les fonctionnalités internes vérifiés sont : 
\begin{itemize}
\item Récupération des paramètres des blocs
	\begin{itemize}
	\item Gain (Gain)
	\item DelayLength (Delay)
	\item Inputs (Sum)
	\end{itemize}
\item Traitement des sous-systèmes (SubSystem)
\item Groupement des additions 
\item Établissement de l'équation pour chaque bloc
\end{itemize}
\bigskip
\hspace{2cm}\textbf{\textcolor{red}{On a les mêmes matrices à la sortie}}  \\
\end{frame}


\begin{frame} \FT{Validation de la génération de la liste oSoP }
	On a \textbf{automatisé} la génération de la liste d'oSoP
	\begin{center}
	\FIGW{1.0}{losop_all.png}
	\end{center}
\end{frame}

\begin{frame} \FT{ Validation du module SIF2VHDL}
	\begin{center}
	\FIGW{0.6}{vhd_valid}
	\end{center}
	\hspace{2cm} Les 2 codes génèrent les même sorties
\end{frame}

\begin{frame} \FT{ Exemple de génération de circuit}
	\begin{center}
	\FIGW{1.0}{circuit}
	\end{center}
	\hspace{2cm} Validation visuelle du circuit
\end{frame}
