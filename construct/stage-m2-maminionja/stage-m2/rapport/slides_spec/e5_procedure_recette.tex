%==============================================================================
\section{Procédure de recette}
%==============================================================================

\begin{frame} \FT{Procédure de recette}
    \BI
	\o Au niveau \textbf{développement}~:\\
	 utiliser des tests unitaires.
	Tous les tests devront passer
	\smallskip
	\o Au niveau des \textbf{fonctionnalités}~:\\ On testera par des exemples déjà traités par les 2 équipes
	\smallskip
	\o Au niveau de la \textbf{chaine complète}~:\\
	\smallskip
	\begin{center}
	\FIGW{0.6}{tool_valid}
	\end{center} 
    \EI
	\note{
	validation chaine complète:A partir d'un diagramme de blocs en simulink, on devrait avoir un code VHDL correspondant qui réalise la même opération en virgule fixe.
	Les mêmes jeux de test sur la simulation Simulink et la simulation du code vhdl issu de l'outil devrait donner à peut près le même résultat.
	}
\end{frame} 
%_________________________________________
\begin{frame} \FT{Procédure de recette}
Pour la partie \textbf{diagramme Simulink vers SIF}~:
	\smallskip
	\BI
	\o Essentiellement \textbf{par comparaison} au résultat d'interprétation manuelle
	\o Comparaison au résultat pour le filtre LWDF déjà été fait "manuellement" par un membre de l'équipe PEQUAN\\
		On doit avoir le \textbf{même résultat}
	\o Petits diagrammes simulink  pour tester les fonctionnalités internes
	\EI
\note{
	un example parametrable en matlab:LWDF 
	}
\end{frame}
%________________________________________
\begin{frame} \FT{Procédure de recette}
Pour la partie utilisant Stratus~: \textbf{oSoP vers VHDL}
	\smallskip
	\BI
	\o FiPoGen peut générer un \textbf{code C} virgule fixe correspondant à l'oSoP
	\o Nous, on génère du code \textbf{VHDL} à partir de l'oSoP
	\o On verifiera par \textbf{double simulation}~:\\
		On génère aléatoirement les mêmes stimulus pour les 2 codes puis on lance les 2 simulations, ce qui devrait nous donner le même résultat.
	\EI
\end{frame} 

%_________________________________________________________________________________________________________________________________________________
% Au niveau de la chaine complète:
% A partir d'un diagramme de blocs en simulink, on devrait avoir un code VHDL correspondant qui réalise la même opération en virgule fixe.
% Si le temps le permet (on aura le flot complet), on verifiera par "co-simulation" avec Matlab. si tout est OK, on devrait avoir le même résultat 
% de simulation à une petite erreur prêt.


% Au niveau des fonctionnalités
% On testera avec des exemples déjà traités par les 2 équipes ou des ex bien connus.
% 
% Pour la partie diagramme Simulink vers SIF:
% La validation se fait essentiellement par comparaison au résultat d'interprétation manuelle.
% un example parametrable en matlab(LWDF) a déjà été fait "manuellement" par un membre de l'équipe PEQUAN
% On doit avoir le même résultat 
% 
% Le test des fonctionnalités internes se fait par des petits diagrammes simulink
%
%
% Pour la partie Stratus: oSoP vers VHDL
% FiPoGen peut générer un code C virgule fixe correspondant à l'oSoP
% Nous, on génère du code VHDL à partir de l'oSoP
% donc, on verifiera par co-simulation 
% On génère aléatoirement les mêmes stimulus pour les 2 codes puis on lance les 2 simulations, ce qui devrait nous donner le même résultat.
%
% On pourra aussi verifier manuellement en inspectant les netlists, pour des diagrammes assez petits

% Au niveau du développement
% Ecrire des tests unitaires pour les méthodes essentiels.
% ( Le Génie logiciel encourage de faire des tests unitaires, même pour un petit projet)
% Tous les tests devront passer.



 
