%==============================================================================
\section{Comment je vais résoudre le problème?}
%==============================================================================

\begin{frame} \FT{Principe de la solution}
	Pour le module \textbf{Simulink vers SIF} :  
    \BI
		\o SLX : conteneur de fichiers XML et non XML, conforme à la norme OPC(Open Packanging Convention)
		\smallskip
		\begin{center}
		\FIGW{0.8}{slx}
		\end{center}	 
	\EI
	\note{
	}
\end{frame}

\begin{frame} \FT{Principe de la solution}
	
		\BI
		\o La forme implicite spécialisé (SIF) : 
		\bigskip
		\begin{center}
		\FIGW{0.6}{sif-all}
		\end{center}	 

		\o Parcourir les blocs (t, x, y) 
		\o Exprimer l'équation pour chaque bloc en f(t, x u)
		\o Aligner les équations et identifier les matrices
		\o Vérification :  Faire à la main puis comparer
		\EI
\end{frame}

\begin{frame} \FT{Principe de la solution}
	Pour le module \textbf{oSoP vers VHDL} 
		\smallskip
		\begin{center}
		\FIGW{1.0}{osop2vhd-all}
		\end{center}
	\BI
		\o \'Etudier l'implémentation des oSoP dans FiPoGen
		\o Comprendre en détails les générateurs de Startus utiles pour notre cas
		\o Comprendre l'outil Stratus (le modèle objet associé) pour développer
	\EI
\end{frame}


