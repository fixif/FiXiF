\begin{command}[@FWR/plus]{plus}
	\desc{Purpose}
add two FWR object (put them in parallel)
	\desc{Syntax}
\matlab{R = R1+R2                  }\\
\matlab{R = plus(R1,R2,generalform)}
	\desc{Parameters}
		\begin{tabular}{l@{\ :\ }p{9cm}}
\matlab{R} &  FWR result                                                                \\
\matlab{R1} &  first FWR                                                                \\
\matlab{R2} &  second FWR                                                               \\
\matlab{generalform} &  (default is true) to use the general form (or a particular form)\\
		\end{tabular}
	\desc{Description}
\fig[scale=0.3]{parallel}{Two realizations in parallel}
Put two realization in parallel (see figure \ref{fig:parallel}).\\
We consider two realizations $\mathcal{R}_1:=(J_1,K_1,L_1,M_1,N_1,P_1,Q_1,R_1,S_1)$
and $\mathcal{R}_2:=(J_2,K_2,L_2,M_2,N_2,P_2,Q_2,R_2,S_2)$ (with compatible size, \I{i.e.}
$m_1=m_2$ and $p_1=p_2$).\\
By introducing the intermediate variables $T_1'$ and $T_2'$ equal to the output
of the two realizations, the resulting realization $\mathcal{R}$ (\I{general} form)
can be expressed in the implicit form by
\begin{footnotesize}
\begin{equation*}
\begin{pmat}({...|.|})
J_1 & 0 & 0 &0 & 0 & 0 & 0 \cr
-L_1 & I_{p_1} & 0 & 0 & 0 &0 & 0 \cr
0 & 0 & J_2 & 0 & 0 &0 & 0 \cr
0 & 0 & -L_2 & I_{p_2} & 0 & 0 & 0 \cr\-
-K_1 & 0 & 0 & 0 & I_{n_1} & 0 & 0 \cr
0 & 0 & -K_2 & 0 & 0 & I_{n_2} & 0 \cr\-
0 & -I_{p} & 0 & -I_{p} & 0 & 0 & I_{p} \cr
\end{pmat}
\begin{pmatrix}
T_1(k+1) \\
T_1'(k+1) \\
T_2(k+1) \\
T_2'(k+1) \\
X_1(k+1) \\
X_2(k+1) \\
Y_2(k)
\end{pmatrix}
=
\begin{pmat}({...|.|})
0 & 0 & 0 & 0 & M_1 & 0 & N_1 \cr
0 & 0 & 0 & 0 & R_1 & 0 & S_1 \cr
0 & 0 & 0 & 0 & 0 & M_2 & N_2 \cr
0 & 0 & 0 & 0 & 0 & R_2 & S_2 \cr\-
0 & 0 & 0 & 0 & P_1 & 0 & Q_1 \cr
0 & 0 & 0 & 0 & 0 & P_2 & Q_2 \cr\-
0 & 0 & 0 & 0 & 0 & 0 & 0 \cr
\end{pmat}
\begin{pmatrix}
T_1(k) \\
T_1'(k) \\
T_2(k) \\
T_2'(k) \\
X_1(k) \\
X_2(k) \\
U_1(k)
\end{pmatrix}
\end{equation*}
\end{footnotesize}
If we allow to regroup $S_1$ and $S_2$ in one term $S$ (and changing a bit
the parametrization if $S_1$ and $S_2$ are both non-zero), the resulting
realization can be expressed in a \I{compact} form :
\begin{footnotesize}
\begin{equation*}
\begin{pmat}({.|.|})
J_1 & 0  & 0 & 0 & 0 \cr
0 &  J_2  & 0 &0 & 0 \cr\-
-K_1 & 0 & I_{n_1} & 0 & 0 \cr
0 & -K_2 & 0 & I_{n_2} & 0 \cr\-
-L_1 & -L_2 & 0 & 0 & I_{p} \cr
\end{pmat}
\begin{pmatrix}
T_1(k+1) \\
T_2(k+1) \\
X_1(k+1) \\
X_2(k+1) \\
Y_2(k)
\end{pmatrix}
=
\begin{pmat}({.|.|})
0 & 0 & M_1 & 0 & N_1 \cr
0 & 0 & 0 & M_2 & N_2 \cr\-
0 & 0 & P_1 & 0 & Q_1 \cr
0 & 0 & 0 & P_2 & Q_2 \cr\-
0 & 0 & R_1 & R_2 & \pa{S_1+S_2} \cr
\end{pmat}
\begin{pmatrix}
T_1(k) \\
T_1'(k) \\
T_2(k) \\
T_2'(k) \\
X_1(k) \\
X_2(k) \\
U_1(k)
\end{pmatrix}
\end{equation*}
\end{footnotesize}
If $S_1$ and/or $S_2$ are null, the two forms are equivalent (in finite precision).
\desc{See also}
\funcName[@FWR/mtimes]{mtimes}
\end{command}


