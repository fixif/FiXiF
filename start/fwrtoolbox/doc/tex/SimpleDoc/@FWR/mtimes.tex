\begin{command}[@FWR/mtimes]{mtimes}
	\desc{Purpose}
Multiply two FWR (put them in cascade)
	\desc{Syntax}
\matlab{R = R1*R2        }\\
\matlab{R = mtimes(R1,R2)}
	\desc{Parameters}
		\begin{tabular}{l@{\ :\ }p{9cm}}
\matlab{R} &  FWR result           \\
\matlab{R1} &  first operand (FWR) \\
\matlab{R2} &  second operand (FWR)\\
		\end{tabular}
	\desc{Description}
\fig[scale=0.3]{cascade}{Two realizations in cascade}
Put two realization in cascade (see figure \ref{fig:cascade}).\\
We consider two realizations $\mathcal{R}_1:=(J_1,K_1,L_1,M_1,N_1,P_1,Q_1,R_1,S_1)$
and $\mathcal{R}_2:=(J_2,K_2,L_2,M_2,N_2,P_2,Q_2,R_2,S_2)$ (with compatible size,
\I{i.e.} $p_1=m_2$).\\
By introducing an intermediate variable $T$ equal to the output of $\mathcal{R}_1$,
the resulting realization $\mathcal{R}$ can be expressed in the implicit form by :
\begin{footnotesize}
\begin{equation*}
\begin{pmat}({..|.|})
J_1 & 0 & 0 & 0 & 0 & 0 \cr
-L_1 & I & 0 & 0 &0 & 0 \cr
0 & -N_2 & J_2 & 0 & 0 & 0 \cr\-
-K_1 & 0 & 0 & I & 0 & 0 \cr
0 & -Q_2 & -K_2 & 0 & I & 0 \cr\-
0 & -S_2 & -L_2 & 0 & 0 & I \cr
\end{pmat}
\begin{pmatrix}
T_1(k+1) \\
T(k+1) \\
T_2(k+1) \\
X_1(k+1) \\
X_2(k+1) \\
Y_2(k)
\end{pmatrix}
=
\begin{pmat}({..|.|})
0 & 0 & 0 & M_1 & 0 & N_1 \cr
0 & 0 & 0 & R_1 & 0 & S_1 \cr
0 & 0 & 0 & 0 & M_2 & 0 \cr\-
0 & 0 & 0 & P_1 & 0 & Q_1 \cr
0 & 0 & 0 & 0 & P_2 & 0 \cr\-
0 & 0 & 0 & 0 & R_2 & 0 \cr
\end{pmat}
\begin{pmatrix}
T_1(k) \\
T(k) \\
T_2(k) \\
X_1(k) \\
X_2(k) \\
U_1(k)
\end{pmatrix}
\end{equation*}
\end{footnotesize}
\desc{See also}
\funcName[@FWR/plus]{plus}
\end{command}


