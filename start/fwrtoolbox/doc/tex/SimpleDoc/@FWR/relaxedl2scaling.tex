\begin{command}[@FWR/relaxedl2scaling]{relaxedl2scaling}
	\desc{Purpose}
Perform a relaxed-$L_2$-scaling on the FWR.
The wordlength are deduced from the FPIS if it is defined.
	\desc{Syntax}
\matlab{R = lrelaxedl2scaling(R,Umax,delta)     }\\
\matlab{[U,Y,W] = relaxedl2scaling(R,Umax,delta)}
	\desc{Parameters}
		\begin{tabular}{l@{\ :\ }p{9cm}}
\matlab{R} &  FWR object                                               \\
\matlab{U,Y,W} &  transformation matrices applied on R                 \\
\matlab{Umax} &  input maximum magnitude (default: Umax = a power of 2)\\
\matlab{delta} &  security parameter (default: delta=1)                \\
		\end{tabular}
	\desc{Description}
Perform a relaxed-$L_2$-scaling.\\
The scaling forces the transfer functions from the inputs to the states and the intermediates variables to have a $L_2$-norm between 1 and 4. Theses norms are given by the diagonal terms of $W_c$ and $J^{-1}\pa{NN^\top+MW_cM^\top}J^{-\top}$.\\
Denote $W_{cX}$ the controllability gramian of the realization ($W_c$) and $W_{cT}$ the controllability gramian related to the intermediate variables. It is given by
\begin{equation}
W_{cX} = \sqrt{\pa{J^{-1}\pa{NN^\top+M W_{cX} M^\top}J^{-\top}}}
\end{equation}
The relaxed-$L_2$-scaling is a transformation that make the realization satisfy the constraints ($\forall 1 \leq i \leq n$)
\begin{eqnarray}
\frac{2^{2{\alpha_X}_i}}{\delta^2} \leq &\pa{W_{cX}}_{i,i}& < 4 \frac{2^{2{\alpha_X}_i}}{\delta^2} \\
\frac{2^{2{\alpha_T}_i}}{\delta^2} \leq &\pa{W_{cT}}_{i,i}& < 4 \frac{2^{2{\alpha_T}_i}}{\delta^2}
\end{eqnarray}
where
\begin{eqnarray}
{\alpha_X}_i &\triangleq& \beta_{X_i}-\beta_U-\mathscr{F}_2\pa{\overset{\max}{U}} \\
{\alpha_T}_i &\triangleq& \beta_{T_i}-\beta_U-\mathscr{F}_2\pa{\overset{\max}{U}}
\end{eqnarray}
and $\mathscr{F}_2(x)$ is defined as the fractional value of $\log_2(x)$:
\begin{equation}
\mathscr{F}_2(x)\triangleq \log_2(x) - \left\lfloor \log_2(x) \right\rfloor
\end{equation}
If the wordlength $\beta_X$ and $\beta_T$ are not defined by the FPIS, they are also supposed to be equal. Moreover, if $\delta=1$ (default case) and $\overset{\max}{U}$ is
a power of $2$, then the realization satisfies the following constraints
\begin{equation}
1 \leq (W_{cX})_{ii} < 4, \hspace{1cm} 1 \leq (W_{cT})_{ii} < 4, \hspace{1cm} \forall 1\leq i \leq n
\end{equation}
The $L_2$-scaling is achieved by a $\mt{U}\mt{Y}\mt{W}$-transformation where $\mt{U}$ and $\mt{W}$ are diagonal with:
\begin{eqnarray}
\pa{\mt{U}}_{ii} &=& \delta \sqrt{\pa{W_{cX}}_{i,i}} 2^{-\mathscr{F}_2(\delta\sqrt{\pa{W_{cX}}_{i,i}})-{\alpha_X}_i} \\
\pa{\mt{W}}_{ii} &=& \delta \sqrt{\pa{W_{cT}}_{i,i}} 2^{-\mathscr{F}_2(\delta\sqrt{\pa{W_{cT}}_{i,i}})-{\alpha_T}_i}
\end{eqnarray}
\desc{See also}
\funcName[@FWR/l2scaling]{l2scaling}
	\desc{References}
\cite{Feng09a} Y.�Feng, P.�Chevrel, and T.�Hilaire. A practical strategy of an efficient and sparse fwl implementation of lti filters. In submitted to ECC?09, 2009.\\
\cite{Hila09a}	T.�Hilaire. Low parametric sensitivity realizations with relaxed l2-dynamic-range-scaling constraints. submitted to IEEE Trans. on Circuits \& Systems II, 2009.
\end{command}


