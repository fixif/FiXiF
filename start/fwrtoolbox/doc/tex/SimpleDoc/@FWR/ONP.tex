\begin{command}[@FWR/ONP]{ONP}
	\desc{Purpose}
Compute the Output Noise Power for a FWR object with Roundoff Before
Multiplication (RBM) computational scheme
	\desc{Syntax}
\matlab{M = ONP( R, roundingMode)}
	\desc{Parameters}
		\begin{tabular}{l@{\ :\ }p{9cm}}
\matlab{M} &  output noise power measure                                           \\
\matlab{R} &  FWR object                                                           \\
\matlab{mode} &  indicates the truncation mode: \matlab{'truncation'} (default) or \matlab{'nearest'}\\
		\end{tabular}
	\desc{Description}
This function computes the Output Noise Power.\\
Let us consider a realization $\mathcal{R}$ described with the implicit form�\eqref{eq:def_implicit}, with transfer function $H$.
When implemented, the steps (i) to (iii) are modified by the add of noises $\xi_T(k)$, $\xi_X(k)$ and $\xi_Y(k)$:
\begin{equation}\label{eq:ONP:implemented_sys}
\begin{array}{r>{\hspace{-3mm}}c<{\hspace{-3mm}}l}
J.T(k+1)  &\leftarrow&  M.X(k) + N.U(k) + \xi_T(k) \\
X(k+1)  &\leftarrow&  K.T(k+1) + P.X(k) + Q.U(k) + \xi_X(k) \\
Y(k)  &\leftarrow&  L.T(k+1) + R.X(k) + S.U(k) + \xi_Y(k)
\end{array}
\end{equation}
%(the noise $J^{-1}\xi_T(k)$ is added on $T(k+1)$).\\
These noises added depend on:
\begin{itemize}
\item the way the computations are organized (the order of the sums) and done,
\item the fixed-point representation of the inputs, the outputs,
\item and the fixed-point representation chosen for the states, the intermediate variables and the coefficients.
\end{itemize}
%They are determined by their first ($\mu$) and second ($\Psi$, $\sigma$) order moments.
They are modeled as independent white noise, characterized by their first and second order moments.
Denote $\xi$ the vector with all the added noise sources:
\begin{equation}
\xi(k) \triangleq \begin{pmatrix} \xi_T(k) \\ \xi_X(k) \\ \xi_Y(k) \end{pmatrix}
\end{equation}
\begin{proposition}
It is then possible to express the implemented system as the initial system with a noise $\xi'(k)$ added on the output(s) (see figure \ref{fig:equivalent_system}).
\fig[scale=0.4]{equivalent_system}{Equivalent system, with noises extracted}\\
$\xi'(k)$ is the noise $\xi(k)$ through the transfer function $H_\xi$ defined by:
\begin{equation}
H_\xi : z \to C_Z \pa{ zI_{n}-A_Z}^{-1} M_1 + M_2
\end{equation}
with
\begin{eqnarray}
M_1 &\triangleq  \begin{pmatrix} KJ^{-1} & I_n & 0 \end{pmatrix} \\
M_2 &\triangleq  \begin{pmatrix} LJ^{-1} & 0 & I_{p_2} \end{pmatrix}
\end{eqnarray}
\end{proposition}
The Output Noise Power is defined as the power of the noises added on the output
\begin{equation}
P \triangleq E{ \xi'(k)\xi'(k)^\top }
\end{equation}
where the $E{.}$ is the mean operator.\\
It is evaluated by \cite{Hila08c}:
\begin{equation}\label{eq:RNP}
P = tr\pa{ \psi_\xi \pa{ M_2^\top M_2 + M_1^\top W_o M_1 } } +  \mu_{\xi'}^\top \mu_{\xi'}
\end{equation}
where $\mu_{\xi'} = (C_Z(I-A_Z)^{-1}M_1+M_2)\mu_\xi$ and $W_o$ is the observability gramian of the system $\mathcal{R}$.
Then, in \I{Roundoff Before Multiplication} (RBM) scheme, the quantizations only occur at the end of the additions, when the accumulator result is stored in intermediate variables, states or output, and a right-shift of $d_{ADD}$ bits is applied.
The lemma \ref{prop:quantization_noise} recalls the noise produced during shift:
\begin{lemma}\label{prop:quantization_noise}
Let $x(k)$ be a signal with fixed-point format $(\beta+d,\alpha+d)$. Right shifting $x(k)$ of $d$ bits is similar to add to $x(k)$ the independent white noise $e(k)$.\\% (see figure \ref{fig:quantization_noise}).
%	\fig[scale=0.4]{quantization_noise}{Quantized a signal is similar to add a noise}\\
The right shift could round $x(k)$ towards $-\infty$ (truncation: default behaviour) or toward the nearest integer (nearest rounding: possible with some additional hardware/software operations \cite{Laps96}). If $d>0$, the moments of $e(k)$ are given by:
\begin{equation}
\begin{array}{|c|c|c|}
\hline & \text{truncation} & \text{best roundoff}\\
\hline \mu_e & 2^{-\gamma-1}(1-2^{-d}) & 2^{-\gamma-d-1} \\
\hline \sigma_e^2 & \frac{2^{-2\gamma}}{12}(1-2^{-2d}) & \frac{2^{-2\gamma}}{12}(1-2^{-2d}) \\
\hline
\end{array}
\end{equation}
else ($d\leq0$) $e(k)$ is null.
\end{lemma}
It is now possible to define the moments of $\xi(k)$:
Denote $\bar{\gamma} \triangleq \begin{pmatrix} \gamma_T \\ \gamma_X \\ \gamma_Y \end{pmatrix}$ and define $s$ by
\begin{equation}
s_i \triangleq \begin{cases}
1 & \text{if } d_{ADDi} > 0 \\
0 & \text{otherwise}
\end{cases}
\end{equation}
Then $\mu_\xi$ is given by:
\begin{equation}
\pa{\mu_\xi}_{i} = \begin{cases}
s_i 2^{-\bar{\gamma}_i-1}  & \text{truncation}\\
s_i 2^{-\bar{\gamma}_i-1-d_{ADD}} & \text{nearest rounding}
\end{cases}
\end{equation}
and, since these noises are independent, $\psi_\xi$ is diagonal with:
\begin{equation}
\pa{\psi_\xi}_{i,i} = s_i\frac{2^{-2\bar{\gamma}_i}}{12} \pa{ 1 - 2^{-d_{ADD}}  }
\end{equation}
\desc{See also}
\funcName[@FWR/setFPIS]{setFPIS}, \funcName[@FWR/RNG]{RNG}
	\desc{References}
\cite{Hila08c} T.�Hilaire, D.�M�nard, and O.�Sentieys. Bit
accurate roundoff noise analysis of fixed-point linear controllers. In Proc. IEEE International Symposium on Computer-Aided Control System Design (CACSD'08), September 2008.
\end{command}


