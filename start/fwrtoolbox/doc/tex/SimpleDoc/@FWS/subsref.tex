\begin{command}[@FWS/subsref]{subsref}
	\desc{Purpose}
Subscripted reference for FWS object.
Here, \matlab{'S.prop'} is equivalent to \matlab{'get(S,\matlab{prop})'}
	\desc{Syntax}
\matlab{value = subsref(S,Sub)}
	\desc{Parameters}
		\begin{tabular}{l@{\ :\ }p{9cm}}
\matlab{value} &  returned value        \\
\matlab{S} &  FWS object                \\
\matlab{Sub} &  layers of subreferencing\\
		\end{tabular}
	\desc{Description}
These functions are called internally when operators \matlab{[]}, \matlab{()}
and \matlab{.} are applied on a FWR object.\\
Only the operator \matlab{.} is valid, and links to \funcName[@FWS/set]{set} and \funcName[@FWS/get]{get}
functions. The command \matlab{S.field} returns the field \matlab{field} of \matlab{S}
(internally, \matlab{get(S,'field')} is called), and \matlab{S.field=vaue} set the field
\matlab{field} of \matlab{S}
	\desc{Example}
\matlab{S.Rini = R}\\
\matlab{S.R}\\
\matlab{S.R.Z(3,3) = 0;}\\
\matlab{S.Rini.WZ = zeros(n);}
\desc{See also}
\funcName[@FWS/set]{set}, \funcName[@FWS/get]{get}, \funcName[@FWS/subsasgn]{subsasgn}, \funcName[@FWR/subsref]{subsref}
\end{command}


