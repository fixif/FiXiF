\chapter{Génération de la Forme Implicite Spécialisé à partir du diagramme de Blocs Simulink}

\section{L'idée}
L'objectif ici est de déduire les 9 matrices (J, K, L, M, N, P, Q, R, S) du S.I.F à partir d'un schéma-blocs représentant un système linéaire, de manière automatique.

\section{Le diagramme de bloc Matlab/Simulink}
Un diagramme de bloc ou schéma-bloc est une représentation graphique simplifié d'un système plus ou moins complexe. Il est composé de blocs connectés par des lignes qui montre l'interaction entre-eux.
Il permet d'exprimer la structure et le flux d'un système pour juste montrer le concept sans entrer dans les détails d'implémentation.

[exemple de schéma-bloc]


Dans Simulink, on a 2 formats pour stocker les schéma-blocs:
MDL : ancienne format
SLX : nouvelle format de fichier, à partir de la r2012a de Matlab


Av : C'est un standard dans l'industrie.
Pb: C'est un nouveau format très peu documenté.


\section{Méthode pour déduire le SIF à partir d'un Diagramme de Blocs}
[ faire la liste en diagramme]
 - Décompresser le fichier SLX pour extraire le fichier \emph{blocdiagram.xml}
 - Parser le fichier pour extraire la liste des blocs et la liste des lignes ( qui indiques les inter-connexions)
 - Aplatir le design s'il y a des sous-systèmes
 - Identifier les blocs en entrée de chaque bloc, cela donne l'équation au niveau de chaque bloc
 - regrouper si possible les blocs sommes 
 - Identifier les équations avec celles du SIF et donc les coefficients formeront les matrices du SIF
 - Rendre triangulaire inférieure la matrice J et réorganiser les matrices concernées : M, N, K, L
 
Explications:
Ce qu'on a:
\begin{itemize}
\item Chaque bloc simulink est identifié par un numéro unique appelé \textbf{\emph{SID}}
\end{itemize}


Ce que l'on met comme hypothèse:
\begin{itemize}
\item chaque bloc est labellisé par $t, x, u, y$ suivit du SID [example] 
\end{itemize}



\section{Implémentation - le module SLX2SIF}

Il suffit de lui donner un fichier Simulink au format SLX, contenant le schéma-bloc pour un système linéaire, et il donne en sortie les matrices du SIF avec les variables liées $(t, x, u, y)$.

[schéma]

\subsection{Les caractéristiques du module}
Pour l'état actuel de notre module, on peut traiter 6 types de blocs Simulinnk:
\begin{itemize}
\item Gain
\item Delay
\item Sum
\item SubSystem
\item Inport
\item Outport
\end{itemize}

[ tableau : blocs - attributs ]

Il peut traiter :
 - des systèmes MIMO
 - les sous-systèmes
