%==============================================================================
\section{Mise en \oe{}uvre}
%==============================================================================
\hspace*{2cm}
\frame{\tableofcontents[currentsection]}
%________________________________________________________________________________________
\begin{frame} \FT{Environnement de dev et d'expérimentation}
\begin{itemize}
\item Gnu/\textbf{Linux} 2.6
\item \textbf{Python 2.7} avec zipfile, lxml.etree, math, scipy.io, numpy, pickle, shutil  
\item FiPoGen, AMPL/Bonmin
\item Stratus (lib arith, modules stratus, packages stratus)
\item \textbf{Git} pour la gestion de version
\item Matlab/Simulink
\item GHDL pour la simulation VHDL
\item Quartus pour voir la NetList RTL
\end{itemize}
\end{frame}

%____________________________________________________________
\begin{frame} \FT{Le flot à développer }
    \BI
	\o Schéma du flot à développer
	\begin{flushleft}
	\begin{center}
	\FIGW{0.9}{tool}
	\end{center}		
	\end{flushleft}
	\smallskip	
	\o \textbf{les tâches} spécifiques au stage:
		\BI 
		\o Concevoir le \textbf{module "SLX to SIF"}
		%qui permet à FiPoGen d'importer des diagrammes Simulink
		\o Développer le \textbf{module "oSoP to VHDL"} à l'aide de Stratus
		%pour génerer l'architecture matérielle correspondant au graphe de calcul généré par FiPoGen.
		\o \textbf{Intégrer} ces modules avec FiPoGen pour former un outil de conception commun entre les 2 équipes
		\EI
	\EI
\note{
	En addition, on peut exploiter les interactions possibles entre les 2 outils.
}
\end{frame}

%________________________________________________________________________________________
\begin{frame} \FT{Développement du module SLX2SIF }
	\begin{center}
	\only<1>{\FIGW{1.0}{"df20"}}
	\only<2>{\FIGW{1.0}{"df21"}}
	\only<3>{\FIGW{1.0}{"df22"}}
	\only<4>{\FIGW{1.0}{"df23"}}
	\only<5>{\FIGW{1.0}{"df24"}}
	\only<6>{\FIGW{1.0}{"df2"}}
	\end{center}
\begin{itemize}
	\pause
	\item Extraire la liste des blocs et la liste des lignes
	\pause
	\item Aplatir le design s'il y a des sous-systèmes
	\pause
	\item Labelliser chaque blocs
	\pause
	\item Établir l'équation au niveau de chaque bloc, regrouper les sommes
	\pause
	\item Déduire le SIF, rendre la matrice J triangulaire
\end{itemize}
\end{frame}

%________________________________________________________________________________________
\begin{frame} \FT{Développement du module oSoP2VHD}
\begin{itemize}
\item \textbf{Principes : } Parcourir l'oSoP et instancier le générateur Stratus correspondant à chaque opérateur
	\bigskip	
	\begin{center}
	\FIGW{1.0}{osop-vhd-build}
	\end{center}
\end{itemize}
\end{frame}

%________________________________________________________________________________________
\begin{frame} \FT{Intégration : le flot complet}
	\begin{center}
	\FIGW{1.0}{fxp_synth_tool}
	\end{center}
\end{frame}
