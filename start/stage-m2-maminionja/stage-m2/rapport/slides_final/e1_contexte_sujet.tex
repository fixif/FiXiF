%==============================================================================
\section{Contexte et Sujet}
%==============================================================================

\begin{frame} \FT{Pourquoi ce stage ?}
    \BI
    \o Le stage fait partie du projet \textbf{FxPSynthesis} - Action initiative du LIP6
       \BI
       \o Collaboration entre les 2 équipes CIAN et PEQUAN
       \o Développement d'un \textbf{outil commun pour la conception de circuits}
       \o Dans la thématique Systèmes embarqués
       \o Un flot complet - accélerer le développement d'application 
       \EI
	\smallskip
    \o Les participants:
		\BI
		\o \textbf{CIAN} ( Circuits Intégés Analogiques et Numériques) \\ \hspace{1cm}Dpt Système sur Puce
		\o \textbf{PEQUAN} (Performance et Qualité des Algorithmes Numériques) \\ \hspace{1cm}Dpt Calcul Scientifique
		\EI
    \EI
\note{
	Ce stage est lié au projet FxPSynthesis, qui est un projet Action initiative du LIP6.
	Ce projet est le fruit d'une collaboration entre les 2 équipes CIAN et PEQUAN du laboratoire, 
	dans l'objectif de développer un outil commun dans le domaine de la conception de circuits mixtes A/N.
	Il s'inscrit dans une thématique "Systèmes embarqués".
	Ainsi, cet outil pourra être utilisé pour accélerer le developpement d'applications (l'outil gérera un flot complet de conception)
}
\end{frame} 
%_____________________________________________________________________________
\begin{frame} \FT{Le sujet }
    \BI
    \o \textbf{Objectif}: avoir un \textcolor{green}{outil} de synthèse de haut-niveau pour les filtres linéaires, en utilisant l'arithmétique virgule fixe 
	\vspace{0.7cm}
	%\smallskip
	\begin{center}
	\FIGW{0.8}{flot}
	\end{center}			
	\EI
\end{frame}	
\begin{frame} \FT{Le sujet }
	\BI
	    \o \textbf{Etat de l'art}:
		\BI
			\o Des outils de synthèse de filtres existe déjà, ex: FDAtool de Matlab 
			\o Par simulations/raffinements - pas très performants.
		\EI
		\smallskip
		\o \textbf{Le projet} permet de développer un outil performant commun entre les 2 équipes.\\
	\smallskip
	\'A notre disposition, on a:
		\BI 
		\o \textbf{FiPoGen}: un outil qui permet de transformer un algo en du code virgule fixe déjà spécifié 
		\o \textbf{Stratus}: qui permet de décrire des composants matériels à l'aide de ces générateurs paramétrables	
		\EI 
    \EI
\note{
	Le stage consiste à développer un outil permettant la conception haut niveau de circuits utilisant l'arithmétique virgule fixe.
	Le but est de rapprocher et de compléter les 2 outils FiPoGen et Stratus développé respectivement par les 2 équipes PEQUAN et CIAN.
	FipoGen permet de transformer un algo en du code virgule fixe déjà spécifié
	Stratus permet de décrire des composants matériels à l'aide de ces générateurs paramétrables	
	Mis ensemble, les 2 formeront une suite dans un flot de conception.

	(Figure flot de conception)
	
	(Certes, des outils commerciales existent déjà comme DSP System Toolbox (FDAtool), HDL Coder de Matlab, mais ...)
	
	}
\end{frame}
%____________________________________________________________________________________
\begin{frame} \FT{Le sujet }
    \BI
	\o Schéma de l'outil à développer
	\begin{flushleft}
	\begin{center}
	\FIGW{0.9}{tool}
	\end{center}		
	\end{flushleft}
	\smallskip	
	\o \textbf{les tâches} spécifiques au stage:
		\BI 
		\o Concevoir le module "SLX to SIF"
		%qui permet à FiPoGen d'importer des diagrammes Simulink
		\o Développer le module "oSoP to VHDL" à l'aide de Stratus
		%pour génerer l'architecture matérielle correspondant au graphe de calcul généré par FiPoGen.
		\o Intégrer ces modules aux outils existant
		\EI
	\EI
\note{
	En addition, on peut exploiter les interactions possibles entre les 2 outils.
}
\end{frame}

%______________________________________________________________________________________________________________________
 


 
