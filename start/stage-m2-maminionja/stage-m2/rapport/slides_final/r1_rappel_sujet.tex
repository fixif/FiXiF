%==============================================================================
\section{Le Sujet}
%==============================================================================
\begin{frame} \FT{Contexte du stage}
    \BI
    \o Le projet \textbf{FxPSynthesis} - Action initiative du LIP6
       \BI
       \o Collaboration entre 2 équipes du LIP6 : CIAN et PEQUAN
       \o Développement d'un \textbf{outil commun} pour la conception de circuits
       \o \textbf{Un flot complet} - accélérer le développement d'applications 
       \EI
	\smallskip
    \o Les participants:
		\BI
		\o \textbf{CIAN} ( Circuits Intégrés Analogiques et Numériques) \\ \hspace{1cm}Dpt Système sur Puce
		\o \textbf{PEQUAN} (Performance et Qualité des Algorithmes Numériques) \\ \hspace{1cm}Dpt Calcul Scientifique
		\EI
    \EI
\end{frame} 

%_____________________________________________________________________________
\begin{frame} \FT{Du filtre au circuit}
    \BI
    \o \textbf{Objectif}: avoir un \textcolor{green}{outil} de synthèse de haut-niveau pour les filtres linéaires, en utilisant l'arithmétique virgule fixe 
	\vspace{0.7cm}
	%\smallskip
	\begin{center}
	\FIGW{0.8}{flot}
	\end{center}			
	\EI
	\smallskip
	\hspace*{1cm}
	\textbf{FiPoGen }: Fixed Point code Generator \\
	\textbf{Stratus} : Générateurs paramétrables de composants matériels
\end{frame}


%_______________________________________________________________________
\begin{frame} \FT{L'existant au LIP6 }

	\`A notre disposition, on a:
		\BI 
		\o \textbf{FiPoGen}: un outil qui permet de transformer un algorithme de haut niveau en du code virgule fixe déjà spécifié. 
			\BI
				\o développé dans l'équipe PEQUAN
				\o Entrée : la forme implicite spécialisée - S.I.F
				\o Sortie : l'oSoP - graphe de calcul en virgule fixe
			\EI

		\bigskip
		\o \textbf{Stratus}: ensemble de générateurs paramétrables et méthodes qui permet de décrire des composants matériels de manière procédurale.	
			\BI
				\o développé dans l'équipe CIAN
				\o en Python
				\o sortie : VHDL
			\EI
		\EI 

\end{frame}

%_______________________________________________________________________________________
\begin{frame} \FT{La problématique}

	\begin{center}
	\FIGW{0.7}{slx2sif}
	\end{center}
	\pause
    \bigskip
    \begin{center}
	\FIGW{0.7}{osop2vhdl}
	\end{center}
	\pause
    \bigskip	
	\begin{center}
	\FIGW{0.7}{toolb}
	\end{center}

\end{frame}

