%==============================================================================
\section{Analyse des Résultats}
%==============================================================================
\hspace*{2cm}
\frame{\tableofcontents[currentsection]}
%_______________________________________________________
\begin{frame} \FT{Analyse des Résultats}
\begin{table}
\begin{tabular}{l | c }
\textbf{Fonctionnalités} & \textbf{Validation} \\
\hline
Comparaison à un exemple fait manuellement & \textcolor{green}{OK} \\
Test sur différents exemples de petites structures & \textcolor{green}{OK} \\
Test sur un exemple avec plus de 80 blocs & \textcolor{green}{OK} \\
Test sur le LWDF & \textcolor{green}{OK} \\
SLX2SIF & \textcolor{green}{OK}\\
Support des systèmes MIMO & \textcolor{green}{OK} \\
Génération d'oSoP & \textcolor{green}{OK} \\
Génération automatique de la liste des oSoPs & \textcolor{green}{OK} \\
Le module oSoP2VHDL & \textcolor{green}{OK} \\
Génération automatique de tout le circuit & \textcolor{green}{OK} \\
Vérification visuel des netlists & \textcolor{green}{OK} \\
La simulation du circuit entier généré & \textcolor{red}{NOK} \\
\end{tabular}
%\caption{Résultats}
\end{table}

\textbf{Remarque :}
	\begin{itemize}
	\item C'est l'optimisation des largeurs qui prends beaucoup plus de temps
	\item Des cas ou la largeur calculé vaut 0 poseront des problèmes !
	\end{itemize}
	
\end{frame}

%____________________________________________
\begin{frame} \FT{Calendrier}
	%\hspace{4cm} \vspace{1cm}
	\begin{center}
	\FIGW{1.0}{echeancier}
	\end{center}
\end{frame}