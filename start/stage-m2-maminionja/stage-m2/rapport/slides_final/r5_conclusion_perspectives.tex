%==============================================================================
\section{Conclusion et Perspectives}
%==============================================================================
\hspace*{2cm}
\frame{\tableofcontents[currentsection]}
%________________________________________
\begin{frame} \FT{Apport du travail de stage}
\begin{itemize}
 
\item \textbf{Mise à l'épreuve de FiPoGen} et correction de quelques bugs, surtout sur la manipulation des SIF et oSoP
\item Ajout d'import Simulink dans FiPogen : le module \textbf{SLX2SIF}
\item Développement d'un module de \textbf{génération de circuit} à partir d'un oSoP avec Stratus : module \textbf{oSoP2VHD}
\item Un \textbf{flot automatique} : Création d'un preuve de concept pour un outil de synthèse de filtre numérique : \textbf{de la modélisation en diagramme de blocs vers la description matérielle}
\item \textbf{Expérience} en Python, en Filtre numérique, en FxP
\end{itemize}
\end{frame} 
%_________________________________________
\begin{frame} \FT{Perspectives}
\begin{itemize}
\item Améliorer le temps d'\textbf{optimisation des largeurs}, Essayer d'autres solveurs
\item Développer FiPoGen pour traiter complètement les \textbf{systèmes MIMO}
\item Étendre le module d'import Simulink pour supporter \textbf{plus de type de blocs}
\item Créer un module inverse \textbf{SIF vers Simulink}
\item Validation du circuit entier au niveau VHDL
\item Gestion plus fine des erreurs
%\item Étendre le flot pour traiter d'\textbf{autres types de circuits} 
\end{itemize}
\end{frame}

\begin{frame} \FT{Merci beaucoup}
	%\hspace{4cm} \vspace{1cm}
	\begin{center}
	\huge \textbf{Merci de votre aimable attention !}
	\end{center}
	
	\begin{center}
	\FIGW{0.2}{ant.png}
	\end{center}
\end{frame}



 
