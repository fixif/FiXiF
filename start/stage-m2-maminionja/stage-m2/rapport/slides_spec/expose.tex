\documentclass[blue]{beamer}
% color : blue, black, gray,green, yellow, white, lightgray, orange 
%\usetheme{warsaw}

%----------------------------------------------- Packages standards nécessaires
\usepackage[utf8x]{inputenc}   %  pour les accents français
\usepackage{pgf,pgfpages}
%\usepackage[pdftex]{color}
\usepackage{xcolor}
\usepackage{multicol}
\usepackage{graphicx}

%----------------------------------------------------Package Beamer Parametrage
\usetheme{Warsaw}
%\usepackage{beamerthemetree}  % le theme d'affichage
%\beamertemplatetransparentcovereddynamic % pour des listes grisée progressives
\useoutertheme{infolines}
\usenavigationsymbolstemplate{} % pour ne pas avoir les symboles de navigation
%\setbeameroption{show notes on second screen=left} % lorsqu'on gère 2 écrans

%--------------------------------------------------- Quelques raccourcis utiles
\def\FT{\frametitle}
\def\BM{\begin{multicols}}  \def\EM{\end{multicols}}
\def\BI{\begin{itemize}}    \def\EI{\end{itemize}}
\def\BE{\begin{enumerate}}  \def\EE{\end{enumerate}}
\def\BD{\begin{description}}\def\ED{\end{description}}
\def\BQ{\begin{quote}}      \def\EQ{\end{quote}}
\def\BC{\begin{center}}     \def\EC{\end{center}}
\def\o{\item}
\def\FIGW#1#2{\includegraphics[width=#1\textwidth]{images/#2}}
\def\FIGH#1#2{\includegraphics[height=#1\textheight]{images/#2}}



\title[Synthèse de Circuits - FxP]{Synthèse de Circuits utilisant l'Arithmetique Virgule Fixe}
\author{Maminionja Ravoson}
\institute[Master SESI]{Roselyne Chotin-Avot / Thibault Hilaire}
\date{07 Juillet 2015}

\begin{document}
\begin{frame}
	\hspace{1cm}
	\begin{minipage}[t]{0.35\textwidth}
	  	\begin{flushleft} \large
		\FIGW{0.6}{upmc.png}
	  	\end{flushleft}
	\end{minipage}
%
	\begin{minipage}[t]{0.4\textwidth}
  		\begin{flushright} \large
		\FIGW{0.3}{lip6.png}
	  	\end{flushright}
	\end{minipage}	

	\titlepage
 \end{frame}


\hspace*{2cm}
\frame{\tableofcontents}
%==============================================================================
\section{Contexte et Sujet}
%==============================================================================

\begin{frame} \FT{Pourquoi ce stage ?}
    \BI
    \o Le stage fait partie du projet \textbf{FxPSynthesis} - Action initiative du LIP6
       \BI
       \o Collaboration entre les 2 équipes CIAN et PEQUAN
       \o Développement d'un \textbf{outil commun pour la conception de circuits}
       \o Dans la thématique Systèmes embarqués
       \o Un flot complet - accélerer le développement d'application 
       \EI
	\smallskip
    \o Les participants:
		\BI
		\o \textbf{CIAN} ( Circuits Intégés Analogiques et Numériques) \\ \hspace{1cm}Dpt Système sur Puce
		\o \textbf{PEQUAN} (Performance et Qualité des Algorithmes Numériques) \\ \hspace{1cm}Dpt Calcul Scientifique
		\EI
    \EI
\note{
	Ce stage est lié au projet FxPSynthesis, qui est un projet Action initiative du LIP6.
	Ce projet est le fruit d'une collaboration entre les 2 équipes CIAN et PEQUAN du laboratoire, 
	dans l'objectif de développer un outil commun dans le domaine de la conception de circuits mixtes A/N.
	Il s'inscrit dans une thématique "Systèmes embarqués".
	Ainsi, cet outil pourra être utilisé pour accélerer le developpement d'applications (l'outil gérera un flot complet de conception)
}
\end{frame} 
%_____________________________________________________________________________
\begin{frame} \FT{Le sujet }
    \BI
    \o \textbf{Objectif}: avoir un \textcolor{green}{outil} de synthèse de haut-niveau pour les filtres linéaires, en utilisant l'arithmétique virgule fixe 
	\vspace{0.7cm}
	%\smallskip
	\begin{center}
	\FIGW{0.8}{flot}
	\end{center}			
	\EI
\end{frame}	
\begin{frame} \FT{Le sujet }
	\BI
	    \o \textbf{Etat de l'art}:
		\BI
			\o Des outils de synthèse de filtres existe déjà, ex: FDAtool de Matlab 
			\o Par simulations/raffinements - pas très performants.
		\EI
		\smallskip
		\o \textbf{Le projet} permet de développer un outil performant commun entre les 2 équipes.\\
	\smallskip
	\'A notre disposition, on a:
		\BI 
		\o \textbf{FiPoGen}: un outil qui permet de transformer un algo en du code virgule fixe déjà spécifié 
		\o \textbf{Stratus}: qui permet de décrire des composants matériels à l'aide de ces générateurs paramétrables	
		\EI 
    \EI
\note{
	Le stage consiste à développer un outil permettant la conception haut niveau de circuits utilisant l'arithmétique virgule fixe.
	Le but est de rapprocher et de compléter les 2 outils FiPoGen et Stratus développé respectivement par les 2 équipes PEQUAN et CIAN.
	FipoGen permet de transformer un algo en du code virgule fixe déjà spécifié
	Stratus permet de décrire des composants matériels à l'aide de ces générateurs paramétrables	
	Mis ensemble, les 2 formeront une suite dans un flot de conception.

	(Figure flot de conception)
	
	(Certes, des outils commerciales existent déjà comme DSP System Toolbox (FDAtool), HDL Coder de Matlab, mais ...)
	
	}
\end{frame}
%____________________________________________________________________________________
\begin{frame} \FT{Le sujet }
    \BI
	\o Schéma de l'outil à développer
	\begin{flushleft}
	\begin{center}
	\FIGW{0.9}{tool}
	\end{center}		
	\end{flushleft}
	\smallskip	
	\o \textbf{les tâches} spécifiques au stage:
		\BI 
		\o Concevoir le module "SLX to SIF"
		%qui permet à FiPoGen d'importer des diagrammes Simulink
		\o Développer le module "oSoP to VHDL" à l'aide de Stratus
		%pour génerer l'architecture matérielle correspondant au graphe de calcul généré par FiPoGen.
		\o Intégrer ces modules aux outils existant
		\EI
	\EI
\note{
	En addition, on peut exploiter les interactions possibles entre les 2 outils.
}
\end{frame}

%______________________________________________________________________________________________________________________
 


 

%==============================================================================
\section{Le problème à résoudre}
%==============================================================================

%_______________________________________________________________________________________
\begin{frame} \FT{Définition et analyse du problème}
	Le module d'entrée pour FiPoGen (SLX2SIF):
	\begin{center}
	\FIGW{0.7}{slx2sif}
	\end{center}
    \BI
	\o Les fichiers de diagramme simulink ne sont pas bien documentés, donc il faut faire un peu de \textbf{reverse-engineering}
		%(ça prend beaucoup de temps!)
    \o Un minimum de compréhension sur \textbf{l'algèbre linéaire, la théorie des graphes et les filtres numériques}
	\o Comprendre la représentation en \textbf{S.I.F} des systèmes linéaires
	\o Trouver l'\textbf{algorithme} pour la transformation
    \EI
		%[ ici, figure à quoi ressemble un  diag et un sif ]
	\note{
	SIF : Specialized Implicite Form : un forme spécifique plus complet de la représentation d'état des systèmes linéaires
	SLX2SIF : permettra d'interpréter un diagramme de blocs Simulink et de générer le S.I.F.(Specialized Implicite Form) correspondant, une sorte de représentation en espace d'état demandé à l'entrée par FiPoGen.
	}
\end{frame} 
%
%________________________________________________________________________________________
\begin{frame} \FT{Définition et analyse du problème}
	Le module de sortie avec Stratus (oSoP2VHDL):
	\begin{center}
	\FIGW{0.7}{osop2vhdl}
	\end{center}
    \BI
    \o Avoir une notion sur l'\textbf{arithmétique virgule fixe} 
	\o Comprendre un peu l'outil \textbf{FiPoGen}, en particulier la \textbf{classe oSoP} 
	\o Savoir utiliser efficacement les \textbf{Générateurs de Stratus}
	\o Un mécanisme pour \textbf{valider} que le code vhdl généré correspond bien à l'oSoP 
    \EI
		%[ici, figure à quoi ressemble un oSoP + netlist vhdl ]
	\note{
	oSoP : ordered Sum of Product : une somme de produit ordonné, à la sortie de FiPoGen
	oSoP2VHDL: permettra de générer un circuit (VHDL) correspondant au graphe de calcul (SoP : Sum of Product) obtenu par FipoGen
	}
\end{frame}
%
%___________________________________________________________________________________________
\begin{frame} \FT{Définition et analyse du problème}
	L' \textbf{intégration} des modules developpés:\\
	On veut avoir un flot de conception "automatique" fonctionnel
	\begin{center}
	\FIGW{0.7}{toolb}
	\end{center}
	\smallskip
    \BI
	\o Les modules doivent être \textbf{compatible} et forment une chaine uniforme avec l'existant,\textbf{ en Python}
    \o Il sera nécessaire de développer des \textbf{"glue logic"} pour mettre ensemble les modules 
	\o Un \textbf{code réutilisable et bien documenté} pour être utilisé par d'autre à la fin du stage.
    \EI
	 %[ ici, figure de la chaine avec les points d'intervention ]
	\note{
		Ce sera cool si on a ça à la fin !! 	
	}
\end{frame}

%==============================================================================
\section{Comment je vais résoudre le problème?}
%==============================================================================

\begin{frame} \FT{Principe de la solution}
	Pour le module \textbf{Simulink vers SIF} :  
    \BI
		\o SLX : conteneur de fichiers XML et non XML, conforme à la norme OPC(Open Packanging Convention)
		\smallskip
		\begin{center}
		\FIGW{0.8}{slx}
		\end{center}	 
	\EI
	\note{
	}
\end{frame}

\begin{frame} \FT{Principe de la solution}
	
		\BI
		\o La forme implicite spécialisé (SIF) : 
		\bigskip
		\begin{center}
		\FIGW{0.6}{sif-all}
		\end{center}	 

		\o Parcourir les blocs (t, x, y) 
		\o Exprimer l'équation pour chaque bloc en f(t, x u)
		\o Aligner les équations et identifier les matrices
		\o Vérification :  Faire à la main puis comparer
		\EI
\end{frame}

\begin{frame} \FT{Principe de la solution}
	Pour le module \textbf{oSoP vers VHDL} 
		\smallskip
		\begin{center}
		\FIGW{1.0}{osop2vhd-all}
		\end{center}
	\BI
		\o \'Etudier l'implémentation des oSoP dans FiPoGen
		\o Comprendre en détails les générateurs de Startus utiles pour notre cas
		\o Comprendre l'outil Stratus (le modèle objet associé) pour développer
	\EI
\end{frame}



%==============================================================================
\section{Tâches à accomplir}
%==============================================================================

\begin{frame} \FT{Identification des tâches à accomplir }
    \BI
		\o \textbf{Prendre en main} le projet, le sujet du stage
		\o Maitriser l'utilisation des \textbf{outils}, Python en particulier
		\o \'Etudier la structure des \textbf{fichiers} matlab (SLX)
		\o Bien comprendre le \textbf{SIF} et les filtres numériques
		\o \textbf{Développement} SLX vers SIF
		\o Développement oSoP vers VHDL 
		\o Test, \textbf{débogage} et optimisation
		\o \textbf{Intégration}
		\o \textbf{Rédaction} du rapport et des autres documentations
    \EI
\end{frame} 
\begin{frame} \FT{Identification des tâches à accomplir }
Chaque \textbf{développement} est constitué par les étapes suivantes:
	\BI
		\o \'Etude de l'existant
		\o Prototypage
		\o Test et validation
		\o Analyse des résultats et correction
		\o Intégration de la fonctionnalité
	\EI
\end{frame}

%==============================================================================
\section{Procédure de recette}
%==============================================================================

\begin{frame} \FT{Procédure de recette}
    \BI
	\o Au niveau \textbf{développement}~:\\
	 utiliser des tests unitaires.
	Tous les tests devront passer
	\smallskip
	\o Au niveau des \textbf{fonctionnalités}~:\\ On testera par des exemples déjà traités par les 2 équipes
	\smallskip
	\o Au niveau de la \textbf{chaine complète}~:\\
	\smallskip
	\begin{center}
	\FIGW{0.6}{tool_valid}
	\end{center} 
    \EI
	\note{
	validation chaine complète:A partir d'un diagramme de blocs en simulink, on devrait avoir un code VHDL correspondant qui réalise la même opération en virgule fixe.
	Les mêmes jeux de test sur la simulation Simulink et la simulation du code vhdl issu de l'outil devrait donner à peut près le même résultat.
	}
\end{frame} 
%_________________________________________
\begin{frame} \FT{Procédure de recette}
Pour la partie \textbf{diagramme Simulink vers SIF}~:
	\smallskip
	\BI
	\o Essentiellement \textbf{par comparaison} au résultat d'interprétation manuelle
	\o Comparaison au résultat pour le filtre LWDF déjà été fait "manuellement" par un membre de l'équipe PEQUAN\\
		On doit avoir le \textbf{même résultat}
	\o Petits diagrammes simulink  pour tester les fonctionnalités internes
	\EI
\note{
	un example parametrable en matlab:LWDF 
	}
\end{frame}
%________________________________________
\begin{frame} \FT{Procédure de recette}
Pour la partie utilisant Stratus~: \textbf{oSoP vers VHDL}
	\smallskip
	\BI
	\o FiPoGen peut générer un \textbf{code C} virgule fixe correspondant à l'oSoP
	\o Nous, on génère du code \textbf{VHDL} à partir de l'oSoP
	\o On verifiera par \textbf{double simulation}~:\\
		On génère aléatoirement les mêmes stimulus pour les 2 codes puis on lance les 2 simulations, ce qui devrait nous donner le même résultat.
	\EI
\end{frame} 

%_________________________________________________________________________________________________________________________________________________
% Au niveau de la chaine complète:
% A partir d'un diagramme de blocs en simulink, on devrait avoir un code VHDL correspondant qui réalise la même opération en virgule fixe.
% Si le temps le permet (on aura le flot complet), on verifiera par "co-simulation" avec Matlab. si tout est OK, on devrait avoir le même résultat 
% de simulation à une petite erreur prêt.


% Au niveau des fonctionnalités
% On testera avec des exemples déjà traités par les 2 équipes ou des ex bien connus.
% 
% Pour la partie diagramme Simulink vers SIF:
% La validation se fait essentiellement par comparaison au résultat d'interprétation manuelle.
% un example parametrable en matlab(LWDF) a déjà été fait "manuellement" par un membre de l'équipe PEQUAN
% On doit avoir le même résultat 
% 
% Le test des fonctionnalités internes se fait par des petits diagrammes simulink
%
%
% Pour la partie Stratus: oSoP vers VHDL
% FiPoGen peut générer un code C virgule fixe correspondant à l'oSoP
% Nous, on génère du code VHDL à partir de l'oSoP
% donc, on verifiera par co-simulation 
% On génère aléatoirement les mêmes stimulus pour les 2 codes puis on lance les 2 simulations, ce qui devrait nous donner le même résultat.
%
% On pourra aussi verifier manuellement en inspectant les netlists, pour des diagrammes assez petits

% Au niveau du développement
% Ecrire des tests unitaires pour les méthodes essentiels.
% ( Le Génie logiciel encourage de faire des tests unitaires, même pour un petit projet)
% Tous les tests devront passer.



 

%==============================================================================
\section{\'Echéancier}
%==============================================================================

\begin{frame} \FT{Echéancier}
	%\hspace{4cm} \vspace{1cm}
	\begin{center}
	\FIGW{1.0}{echeancier}
	\end{center}
\end{frame}

\end{document}

%_______________________________________________________________________________
% STAGES « RECHERCHE » CONCERNANT LES ELEVES DU MASTER UPMC, spécialité ACSI, M2
%_______________________________________________________________________________
% Le stage « RECHERCHE » se déroule en deux phases :
%
% 1. Phase de Spécification
%   
%    Prise en main du problème, compréhension du sujet, analyse de
%    différentes solutions possibles, choix des outils, identification des
%    tâches à accomplir, organisation du travail, et définition détaillée
%    des objectifs à atteindre et de la procédure de recette.
%
%    Le présent document a pour objectif de présenter le produit de cette phase
%    de spécification en 15 transparents max pour un exposé de 30mn
%
%    La soutenance de spécification et le rapport associé jouent le rôle d'un 
%    cahier des charges :
%    toute modification des objectifs ou de la procédure de recette doit
%    être négociée avec le responsable du sujet et approuvée par le
%    responsable du stage. Les modifications de dernière minute sont
%    irrecevables.
%
% 2. Phase de Réalisation
%
%    Mise en oeuvre de la solution retenue. Mise au point et déboguage.
%    Validation, expérimentation et analyse des résultats.
%
%    L'évaluation en septembre : rapport final + soutenance finale
% 
%    La soutenance finale et le rapport associé reprennent le contenu de la
%    spécification, et la complète en ajoutant :
%    1.      l'analyse des différences entre la réalisation et la spécification,
%    2.      les résultats détaillés des expérimentations et des tests de
%            validation.
%
%    Durée de la soutenance : 30'. L'encadrant doit être présent à la soutenance.
% 
%    Le rapport et la soutenance finale contribuent pour 70% à la note de l'UE.


